\documentclass[a4paper]{article}

\title{Solutions to Problem Set~4}
\author{L. F. \textsc{Jaw}}

\usepackage[T1]{fontenc}
\usepackage{textcomp}
\usepackage{mathtools,amssymb,amsthm}
\usepackage[hmargin=1in,vmargin=1in]{geometry}
\usepackage{graphicx,cancel}
\usepackage{hyperref}
\hypersetup{%
  colorlinks=true,
  urlcolor=[rgb]{0,0.2,0.6},
  linkcolor={.},
  bookmarksdepth=2}
\usepackage{bookmark}

\frenchspacing

\newcommand*{\parasp}{\setlength{\parskip}{10pt plus 2pt minus 3pt}}
\newcommand*{\noparasp}{\setlength{\parskip}{0pt plus 1pt}}
\newcommand*{\setparasp}[1]{\setlength{\parskip}{#1}}
\newcommand*{\pskip}{\vskip 10pt plus 2pt minus 3pt}
% \newcommand\LEFTRIGHT[3]{\left#1 #3 \right#2}
\newcommand\SetSymbol[1][]{%
  \nonscript\:#1\vert
  \allowbreak
  \nonscript\:
  \mathopen{}}
% \newcommand*{\paren}[1]{\LEFTRIGHT(){#1}}
\DeclarePairedDelimiterX{\paren}[1]{\lparen}{\rparen}{%
  \renewcommand{\mid}{\SetSymbol[\delimsize]}#1}
% \newcommand*{\brkt}[1]{\LEFTRIGHT[]{#1}}
\DeclarePairedDelimiterX{\brkt}[1]{\lbrack}{\rbrack}{%
  \renewcommand{\mid}{\SetSymbol[\delimsize]}#1}
\DeclarePairedDelimiterX{\brce}[1]{\lbrace}{\rbrace}{%
  \renewcommand{\mid}{\SetSymbol[\delimsize]}#1}
\newcommand*{\unit}[1]{\,\mathrm{#1}}
\newcommand*{\DeclareUnit}[2]{\newcommand*{#1}{\unit{#2}}}
\DeclareUnit{\cm}{cm}
% \renewcommand*{\m}{\unit{m}}
\DeclareUnit{\m}{m}
\DeclareUnit{\kg}{kg}
\DeclareUnit{\s}{s}
\newcommand*{\R}{\mathbb{R}}
\newcommand*{\Z}{\mathbb{Z}}
% \newcommand*{\Rp}{(0,+\infty)}
% \newcommand*{\Rm}{(-\infty,0)}
\newcommand*{\deduce}{\mathrel{\Downarrow}}
% \newcommand*{\abs}[1]{\left\lvert #1 \right\rvert}
\DeclarePairedDelimiter{\abs}{\lvert}{\rvert}
% \newcommand*{\ceil}[1]{\left\lceil#1\right\rceil}
\DeclarePairedDelimiter{\ceil}{\lceil}{\rceil}
\newcommand*{\textop}[1]{\mathop{\text{#1}}}

\newcommand*{\enumparen}[1]{(\makebox[0.6em][c]{#1})}
\renewcommand{\labelenumii}{\enumparen{\theenumii}}

\DeclareMathOperator{\arccosh}{arccosh}
\DeclareMathOperator{\gammaf}{\Gamma}
\DeclareMathOperator{\var}{var}
\DeclareMathOperator{\Ber}{Bernoulli}
\DeclareMathOperator{\Cov}{Cov}
\DeclareMathOperator{\E}{E}
\def\argmax{\qopname\relax m{arg\,max}}
\DeclarePairedDelimiterXPP{\Eb}[1]{\E}{\lbrack}{\rbrack}{}{%
  \renewcommand{\mid}{\SetSymbol[\delimsize]}#1}
\DeclarePairedDelimiterXPP{\varp}[1]{\var}{\lparen}{\rparen}{}{%
  \renewcommand{\mid}{\SetSymbol[\delimsize]}#1}
\DeclarePairedDelimiterXPP{\Covp}[1]{\Cov}{\lparen}{\rparen}{}{%
  \renewcommand{\mid}{\SetSymbol[\delimsize]}#1}
\DeclarePairedDelimiterXPP{\expp}[1]{\exp}{\lbrace}{\rbrace}{}{#1}
\renewcommand*{\Pr}{\mathop{P}}
\DeclarePairedDelimiterXPP{\Prp}[1]{\Pr}{\lparen}{\rparen}{}{%
  \renewcommand{\mid}{\SetSymbol[\delimsize]}#1}
\newcommand*{\pnorm}{\mathop{\Phi}}
\DeclarePairedDelimiterXPP{\pnormp}[1]{\pnorm}{\lparen}{\rparen}{}{#1}
\newcommand*{\dnorm}{\mathop{\varphi}}
\DeclarePairedDelimiterXPP{\dnormp}[1]{\dnorm}{\lparen}{\rparen}{}{#1}
\newcommand*{\qnorm}{\mathop{\Phi^{-1}}}
%\newcommand*{\diff}{\mathop{}\!d}
\newcommand*{\diff}{\mathop{}\!\mathit{d}}
%\newcommand*{\diff}{\mathop{}\!\mathrm{d}}
\newcommand*{\dx}{\diff x}
\newcommand*{\dy}{\diff y}
\newcommand*{\dz}{\diff z}
\newcommand*{\ds}{\diff s}
\newcommand*{\dt}{\diff t}
\newcommand*{\du}{\diff u}
\newcommand*{\dv}{\diff v}
\newcommand*{\dtheta}{\diff \theta}
\newcommand*{\dd}[2][]{\frac{\diff#1}{\diff#2}}
\newcommand*{\ddx}{\frac{\diff}{\dx}}
\newcommand*{\ddt}{\frac{\diff}{\dt}}
\newcommand*{\ddy}{\dd y}
\newcommand*{\ddtheta}{\frac{\diff}{\dtheta}}
\newcommand*{\ddz}{\dd z}
\newcommand*{\fwdf}{\mathop{}\!\Delta}
\newcommand*{\dydx}{\frac\dy\dx}
\newcommand*{\pdpd}[2][]{\frac{\partial#1}{\partial#2}}
\newcommand*{\pdpdx}{\frac\partial{\partial x}}
\newcommand*{\pdpdy}{\frac\partial{\partial y}}
\newcommand*{\pdpdz}{\frac\partial{\partial z}}
\newcommand*{\pdpdu}{\frac\partial{\partial u}}
\newcommand*{\pdpdv}{\frac\partial{\partial v}}
\newcommand*{\pdpdt}{\frac\partial{\partial t}}
\newcommand*{\pdzpdx}{\frac{\partial z}{\partial x}}
\newcommand*{\pdzpdy}{\frac{\partial z}{\partial y}}
\newcommand*{\pdzpdt}{\frac{\partial z}{\partial t}}
\newcommand*{\pdxpdt}{\frac{\partial x}{\partial t}}
\newcommand*{\pdypdt}{\frac{\partial y}{\partial t}}

% \usepackage[lite,subscriptcorrection,nofontinfo]{mtpro2}
\usepackage{fontspec}

\setmainfont{Palatino Linotype}[Ligatures=TeX,Numbers=OldStyle]
\setmonofont{Source Code Pro}
% \usepackage[integrals]{wasysym}
\usepackage{fontawesome}

\usepackage[math-style=TeX]{unicode-math}
\setmathfont{TeX Gyre Pagella Math}

\usepackage{microtype}

\contitem
\let\reason\text
\let\vect\symbf

\AtBeginDocument{%
  % \renewcommand{\perp}{\mathrel{\bot}}
  \let\leq\leqslant
  \let\le\leq
  \let\geq\geqslant
  \let\ge\geq}


\begin{document}
\maketitle

This is my own solution to Problem Set~4 of
\href{https://ocw.mit.edu/courses/mathematics/18-014-calculus-with-theory-fall-2010/assignments/}{18.014}.
Problem~1 and 6 are taken from Apostol's \textit{Calculus} and Problem~3 is taken from Munkres's Course Note~F (1: 83, 94;).

\begin{enumerate}
\item
  \begin{enumerate}
    \everymath{\displaystyle}
  \item \(\lim_{x \to 0} \frac{\sin 5x}{\sin x} = 5\).
  \item \(\lim_{x \to 0} \frac{\sin 5x - \sin 3x}{x} = 2\).
  \item \(\lim_{x \to 0} \frac{1 - \sqrt{1-x^2}}{x^2} = \frac12\).
  \end{enumerate}

\item Let \(A(x) = \int_{-2}^x f(t) \dt\) where \(f(t) = -1\) if
  \(t < 0\) and \(f(t) = 1\) if \(t \ge 0\).  Graph \(y = A(x)\) for
  \(x \in [-2,2]\).  Using \(\epsilon\)--\(\delta\) language, show that
  \(\lim_{x \to 0} A(x)\) exists and find its value.

\item Let \(f(x)\) be defined for all \(x\), and continuous except for
  \(x = -1\) and \(x = 3\).  Let
  \[
    g(x) =
    \begin{cases}
      x^2 + 1, & \tfor x > 0, \\
      x - 3, & \tfor x \le 0.
    \end{cases}
  \]
  For what values of \(x\) can you be sure that \(f(g(x))\) is continuous?
  Explain.

\item Suppose that \(g\), \(h\) are two continuous functions on \([a,b]\).
  Suppose there exists \(c \in (a,b)\) such that \(g(c) = h(c)\).  Define
  \(f(x)\) such that \(f(x) = g(x)\) for \(x < c\) and \(f(x) = h(x)\) for
  \(x \ge c\).  Prove that \(f\) is continuous on \([a,b]\).

\item Let \(f(x) = \sin(1/x)\) for \(x \in \mathbf{R}\), \(x \ne 0\).
  Show that for any \(a \in \mathbf{R}\), the function \(g(x)\) defined by
  \[
    g(x) =
    \begin{cases}
      f(x), & x \ne 0, \\
      a, & x = 0,
    \end{cases}
  \]
  is not continuous at \(x = 0\).

\item Given a real-valued function \(f\) which is continuous on the closed
  inverval \([0,1]\).  Assume that \(0 \le f(x) \le 1\) for each \(x\) in
  \([0,1]\).  Prove that there is at least one point \(c\) in \([0,1]\)
  for which \(f(c) = c\).  Such a point is called a \emph{fixed point} of
  \(f\).  The result of this exercise is a special case of \emph{Brouwer's
    fixed-point theorem}.

\item[\bonus] Let \(f\) be a bounded function that is integrable on
  \([a,b]\).  Prove that there exists \(c \in \mathbf{R}\) with
  \(a \le c \le b\) such that \(\int_a^b f(x) \dx = 2 \int_a^c f(x) \dx\).
\end{enumerate}
\end{document}
