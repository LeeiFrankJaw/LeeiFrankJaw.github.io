\documentclass{article}

\usepackage[T1]{fontenc}
\usepackage{textcomp}
\usepackage{mathtools,amssymb,amsthm}
\usepackage[b5paper,hmargin=1.25in,vmargin=1in]{geometry}
\usepackage{graphicx,cancel}
% \usepackage{unicode-math}
\usepackage[lite,subscriptcorrection,nofontinfo]{mtpro2}
\usepackage[no-math]{fontspec}
\usepackage[colorlinks=true,allcolors=blue]{hyperref}

\setmainfont{Times New Roman}
% \setmathfont[math-style=TeX]{Cambria Math}
%\usepackage[integrals]{wasysym}

\newcommand*{\parasp}{\setlength{\parskip}{10pt plus 2pt minus 3pt}}
\newcommand*{\noparasp}{\setlength{\parskip}{0pt plus 1pt}}
\newcommand*{\setparasp}[1]{\setlength{\parskip}{#1}}
\newcommand*{\pskip}{\vskip 10pt plus 2pt minus 3pt}
% \newcommand\LEFTRIGHT[3]{\left#1 #3 \right#2}
\newcommand*{\paren}[1]{\LEFTRIGHT(){#1}}
\newcommand*{\brkt}[1]{\LEFTRIGHT[]{#1}}
\newcommand*{\unit}[1]{\,\mathrm{#1}}
\newcommand*{\DeclareUnit}[2]{\newcommand*{#1}{\unit{#2}}}
\DeclareUnit{\cm}{cm}
% \renewcommand*{\m}{\unit{m}}
\DeclareUnit{\m}{m}
\DeclareUnit{\kg}{kg}
\DeclareUnit{\s}{s}
\newcommand*{\R}{\mathbb{R}}
\newcommand*{\Rp}{(0,+\infty)}
\newcommand*{\Rm}{(-\infty,0)}
\newcommand*{\deduce}{\mathrel{\Downarrow}}
\newcommand*{\abs}[1]{\left\lvert #1 \right\rvert}
\newcommand*{\reason}[1]{\langle \, \text{#1} \, \rangle}

\newcounter{ListCounter}
\newenvironment{enumerate*}%
{\begin{list}%
        {\arabic{ListCounter}.}%
        {\usecounter{ListCounter}%
            \setlength{\topsep}{1.5pt}%
            \setlength{\itemsep}{1.5pt} } }%
    {\end{list}}

\DeclareMathOperator{\arccosh}{arccosh}
\DeclareMathOperator{\gammaf}{\Gamma}
%\newcommand*{\diff}{\mathop{}\!d}
\newcommand*{\diff}{\mathop{}\!\mathit{d}}
%\newcommand*{\diff}{\mathop{}\!\mathrm{d}}
\newcommand*{\dx}{\diff x}
\newcommand*{\dy}{\diff y}
\newcommand*{\dt}{\diff t}
\newcommand*{\du}{\diff u}
\newcommand*{\dv}{\diff v}
\newcommand*{\dtheta}{\diff \theta}
\newcommand*{\ddx}{\frac{\diff}{\dx}}
\newcommand*{\fwdf}{\mathop{}\!\Delta}
\newcommand*{\dydx}{\frac\dy\dx}
\newcommand*{\pdpdx}{\frac\partial{\partial x}}
\newcommand*{\pdpdy}{\frac\partial{\partial y}}
\newcommand*{\pdpdz}{\frac\partial{\partial z}}
\newcommand*{\pdpdu}{\frac\partial{\partial u}}
\newcommand*{\pdpdv}{\frac\partial{\partial v}}
\newcommand*{\pdpdt}{\frac\partial{\partial t}}
\newcommand*{\pdzpdx}{\frac{\partial z}{\partial x}}
\newcommand*{\pdzpdy}{\frac{\partial z}{\partial y}}
\newcommand*{\pdzpdt}{\frac{\partial z}{\partial t}}
\newcommand*{\pdxpdt}{\frac{\partial x}{\partial t}}
\newcommand*{\pdypdt}{\frac{\partial y}{\partial t}}

\AtBeginDocument{%
	\renewcommand{\perp}{\mathrel{\bot}}
	\let\leq\leqslant
	\let\le\leq}

\newcommand*{\Prb}[1]{\section*{Problem #1}}
\newcommand{\Q}[1]{\textbf{Question:} #1}
\newcommand*{\A}[1]{\textbf{Answer:} #1}
\newcommand{\Prblm}[3]{\Prb{#1} \Q{#2} \\[6pt] \A{#3}}


\title{\bf Solutions to Recitation~1}
\author{Lei Zhao}
% \date{}

\begin{document}
\maketitle
This is my own solution to the Recitation~1 from the course
\href{https://ocw.mit.edu/courses/mathematics/18-014-calculus-with-theory-fall-2010/recitations/}{18.014},
which includes Exercise~10, 15, and 18 from Apostol's \textit{Calculus}
(1: 16).


\Prblm{1}{Prove \(A \cap (B \cup C) = (A \cap B) \cup (A \cap C)\) and
  \(A \cup (B \cap C) = (A \cup B) \cap (A \cup C)\).} {Let
  \(X = A \cap (B \cup C)\) and \(Y = (A \cap B) \cup (A \cap C)\).
  If \(x \in X\), then \(x \in A\) and \(x \in B \cup C\), which means
  \(x \in B\) or \(x \in C\).  If \(x \in B\), then
  \(x \in A \cap B\), which implies \(x \in Y\).  Similarly if
  \(x \in C\), we can also deduce \(x \in Y\).  In both cases, we have
  \(x \in Y\); hence, \(X \subseteq Y\).  Conversely, if \(x \in Y\),
  then \(x \in A \cap B\) or \(x \in A \cap C\).  If
  \(x \in A \cap B\), then \(x \in A\) and \(x \in B\), which implies
  \(x \in B \cup C\). Thus \(x \in X\).  By the same fasion if
  \(x \in A \cap C\), \(x \in X\).  In both cases, we have
  \(x \in X\).  This is to say \(Y \subseteq X\).  Therefore,
  \(X = Y\). \qed}

{\parasp Let \(X = A \cup (B \cap C)\) and
  \(Y = (A \cup B) \cap (A \cup C)\).  If \(x \in X\), then
  \(x \in A\) or \(x \in B \cap C\).  If \(x \in A\), then
  \(x \in A \cup B\) and \(x \in A \cup C\), which is the same as
  \(x \in Y\).  If \(x \in B \cap C\), then \(x \in B\) and
  \(x \in C\), which further implies that \(x \in B \cup A\) and
  \(x \in C \cup A\); hence \(x \in Y\).  In both cases, we have
  \(x \in Y\); thus \(X \subseteq Y\).  Conversely, if \(x \in Y\),
  then \(x \in A \cup B\) and \(x \in A \cup C\).  If \(x \in A\),
  then \(x \in X\).  If \(x \notin A\), it must be the case that
  \(x \in B\) and \(x \in C\).  This is to say \(x \in B \cap C\),
  which implies \(x \in X\).  In both cases, we have \(x \in X\);
  hence \(Y \subseteq X\).  Therefore, \(X = Y\). \qed}

\Prblm{2}{Prove that if \(A \subseteq C\) and \(B \subseteq C\), then
  \(A \cup B \subseteq C\).}  {If \(x \in A \cup B\), then \(x \in A\)
  or \(x \in B\).  If \(x \in A\), then by \(A \subseteq C\) we have
  \(x \in C\).  Similarly, if \(x \in B\), then by \(B \subseteq C\)
  we have \(x \in C\).  Hence, \(A \cup B \subseteq C\). \qed}


\Prblm{3}{Prove \(A - (B \cap C) = (A-B) \cup (A-C)\).}  {Let
  \(X = A - (B \cap C)\) and \(Y = (A-B) \cup (A-C)\).  If
  \(x \in X\), then \(x \in A\) and \(x \notin B \cap C\), which means
  it is not the case that \(x \in B\) and \(x \in C\).  This is the
  same as saying \(x \notin B\) or \(x \notin C\).  If \(x \notin B\),
  then \(x \in A-B\), which implies \(x \in Y\).  And if
  \(x \notin C\), then \(x \in A-C\), which also implies \(x \in Y\).
  Thus \(X \subseteq Y\).  Conversely, if \(x \in Y\), then
  \(x \in A-B\) or \(x \in A-C\).  If \(x \in A-B\), then \(x \in A\)
  and \(x \notin B\), which implies \(x \notin B \cap C\).  Thus
  \(x \in X\).  Similarly if \(x \in A-C\), we deduces that
  \(x \in X\).  This means \(Y \subseteq X\).  Therefore, \(X =
  Y\). \qed}

\end{document}

% Local Variables:
% TeX-engine: xetex
% End:
