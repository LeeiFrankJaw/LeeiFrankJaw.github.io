\documentclass[a4paper]{article}

\title{Solutions to Problem Set~3}
\author{L. F. \textsc{Jaw}}

\usepackage[T1]{fontenc}
\usepackage{textcomp}
\usepackage{mathtools,amssymb,amsthm}
\usepackage[hmargin=1in,vmargin=1in]{geometry}
\usepackage{graphicx,cancel}
\usepackage{hyperref}
\hypersetup{%
  colorlinks=true,
  urlcolor=[rgb]{0,0.2,0.6},
  linkcolor={.},
  bookmarksdepth=2}
\usepackage{bookmark}

\frenchspacing

\newcommand*{\parasp}{\setlength{\parskip}{10pt plus 2pt minus 3pt}}
\newcommand*{\noparasp}{\setlength{\parskip}{0pt plus 1pt}}
\newcommand*{\setparasp}[1]{\setlength{\parskip}{#1}}
\newcommand*{\pskip}{\vskip 10pt plus 2pt minus 3pt}
% \newcommand\LEFTRIGHT[3]{\left#1 #3 \right#2}
\newcommand\SetSymbol[1][]{%
  \nonscript\:#1\vert
  \allowbreak
  \nonscript\:
  \mathopen{}}
% \newcommand*{\paren}[1]{\LEFTRIGHT(){#1}}
\DeclarePairedDelimiterX{\paren}[1]{\lparen}{\rparen}{%
  \renewcommand{\mid}{\SetSymbol[\delimsize]}#1}
% \newcommand*{\brkt}[1]{\LEFTRIGHT[]{#1}}
\DeclarePairedDelimiterX{\brkt}[1]{\lbrack}{\rbrack}{%
  \renewcommand{\mid}{\SetSymbol[\delimsize]}#1}
\DeclarePairedDelimiterX{\brce}[1]{\lbrace}{\rbrace}{%
  \renewcommand{\mid}{\SetSymbol[\delimsize]}#1}
\newcommand*{\unit}[1]{\,\mathrm{#1}}
\newcommand*{\DeclareUnit}[2]{\newcommand*{#1}{\unit{#2}}}
\DeclareUnit{\cm}{cm}
% \renewcommand*{\m}{\unit{m}}
\DeclareUnit{\m}{m}
\DeclareUnit{\kg}{kg}
\DeclareUnit{\s}{s}
\newcommand*{\R}{\mathbb{R}}
\newcommand*{\Z}{\mathbb{Z}}
% \newcommand*{\Rp}{(0,+\infty)}
% \newcommand*{\Rm}{(-\infty,0)}
\newcommand*{\deduce}{\mathrel{\Downarrow}}
% \newcommand*{\abs}[1]{\left\lvert #1 \right\rvert}
\DeclarePairedDelimiter{\abs}{\lvert}{\rvert}
% \newcommand*{\ceil}[1]{\left\lceil#1\right\rceil}
\DeclarePairedDelimiter{\ceil}{\lceil}{\rceil}
\newcommand*{\textop}[1]{\mathop{\text{#1}}}

\newcommand*{\enumparen}[1]{(\makebox[0.6em][c]{#1})}
\renewcommand{\labelenumii}{\enumparen{\theenumii}}

\DeclareMathOperator{\arccosh}{arccosh}
\DeclareMathOperator{\gammaf}{\Gamma}
\DeclareMathOperator{\var}{var}
\DeclareMathOperator{\Ber}{Bernoulli}
\DeclareMathOperator{\Cov}{Cov}
\DeclareMathOperator{\E}{E}
\def\argmax{\qopname\relax m{arg\,max}}
\DeclarePairedDelimiterXPP{\Eb}[1]{\E}{\lbrack}{\rbrack}{}{%
  \renewcommand{\mid}{\SetSymbol[\delimsize]}#1}
\DeclarePairedDelimiterXPP{\varp}[1]{\var}{\lparen}{\rparen}{}{%
  \renewcommand{\mid}{\SetSymbol[\delimsize]}#1}
\DeclarePairedDelimiterXPP{\Covp}[1]{\Cov}{\lparen}{\rparen}{}{%
  \renewcommand{\mid}{\SetSymbol[\delimsize]}#1}
\DeclarePairedDelimiterXPP{\expp}[1]{\exp}{\lbrace}{\rbrace}{}{#1}
\renewcommand*{\Pr}{\mathop{P}}
\DeclarePairedDelimiterXPP{\Prp}[1]{\Pr}{\lparen}{\rparen}{}{%
  \renewcommand{\mid}{\SetSymbol[\delimsize]}#1}
\newcommand*{\pnorm}{\mathop{\Phi}}
\DeclarePairedDelimiterXPP{\pnormp}[1]{\pnorm}{\lparen}{\rparen}{}{#1}
\newcommand*{\dnorm}{\mathop{\varphi}}
\DeclarePairedDelimiterXPP{\dnormp}[1]{\dnorm}{\lparen}{\rparen}{}{#1}
\newcommand*{\qnorm}{\mathop{\Phi^{-1}}}
%\newcommand*{\diff}{\mathop{}\!d}
\newcommand*{\diff}{\mathop{}\!\mathit{d}}
%\newcommand*{\diff}{\mathop{}\!\mathrm{d}}
\newcommand*{\dx}{\diff x}
\newcommand*{\dy}{\diff y}
\newcommand*{\dz}{\diff z}
\newcommand*{\ds}{\diff s}
\newcommand*{\dt}{\diff t}
\newcommand*{\du}{\diff u}
\newcommand*{\dv}{\diff v}
\newcommand*{\dtheta}{\diff \theta}
\newcommand*{\dd}[2][]{\frac{\diff#1}{\diff#2}}
\newcommand*{\ddx}{\frac{\diff}{\dx}}
\newcommand*{\ddt}{\frac{\diff}{\dt}}
\newcommand*{\ddy}{\dd y}
\newcommand*{\ddtheta}{\frac{\diff}{\dtheta}}
\newcommand*{\ddz}{\dd z}
\newcommand*{\fwdf}{\mathop{}\!\Delta}
\newcommand*{\dydx}{\frac\dy\dx}
\newcommand*{\pdpd}[2][]{\frac{\partial#1}{\partial#2}}
\newcommand*{\pdpdx}{\frac\partial{\partial x}}
\newcommand*{\pdpdy}{\frac\partial{\partial y}}
\newcommand*{\pdpdz}{\frac\partial{\partial z}}
\newcommand*{\pdpdu}{\frac\partial{\partial u}}
\newcommand*{\pdpdv}{\frac\partial{\partial v}}
\newcommand*{\pdpdt}{\frac\partial{\partial t}}
\newcommand*{\pdzpdx}{\frac{\partial z}{\partial x}}
\newcommand*{\pdzpdy}{\frac{\partial z}{\partial y}}
\newcommand*{\pdzpdt}{\frac{\partial z}{\partial t}}
\newcommand*{\pdxpdt}{\frac{\partial x}{\partial t}}
\newcommand*{\pdypdt}{\frac{\partial y}{\partial t}}

% \usepackage[lite,subscriptcorrection,nofontinfo]{mtpro2}
\usepackage{fontspec}

\setmainfont{Palatino Linotype}[Ligatures=TeX,Numbers=OldStyle]
\setmonofont{Source Code Pro}
% \usepackage[integrals]{wasysym}
\usepackage{fontawesome}

\usepackage[math-style=TeX]{unicode-math}
\setmathfont{TeX Gyre Pagella Math}

\usepackage{microtype}


\contitem

\usepackage{tikz}
\usetikzlibrary{positioning,shapes,automata,external}
\tikzexternalize[mode=list and make,prefix=figures/]
% \tikzexternaldisable

\let\reason\text
\let\vect\symbf

\AtBeginDocument{%
  % \renewcommand{\perp}{\mathrel{\bot}}
  \let\leq\leqslant
  \let\le\leq
  \let\geq\geqslant
  \let\ge\geq}


\begin{document}
\maketitle

This is my own solution to Problem Set~3 of
\href{https://ocw.mit.edu/courses/mathematics/18-014-calculus-with-theory-fall-2010/assignments/}{18.014}.
The first two problems are from Apostol's \textit{Calculus} (1: 83, 94).

\begin{enumerate}
\item Find all values of \(c\) for which
  \begin{enumerate}
    \everymath{\displaystyle}
  \item \(\int_0^c x(1-x) \dx = 0,\)
  \item \(\int_0^c \abs[\big]{x(1-x)} \dx = 0.\)
  \end{enumerate}

  \begin{enumerate}
  \item We have
    \begin{displaymath}
      \int_0^c x(1-x) \dx
        = \int_0^c (x - x^2) \dx
        = \frac{c^2}{2} - \frac{c^3}{3}
        = c^2 \paren*{\frac12 - \frac{c}{3}}
        = 0.
    \end{displaymath}
    Solve the above equation and we obtain either \(c = 0\) or \(c = 3/2\).

  \item We have
    \begin{align*}
      \int_0^c \abs[\big]{x(1-x)} \dx
        &= \begin{cases}
          \int_0^c x(1-x) \dx,                       & 0 \le c \le 1, \\[1ex]
          \int_0^1 x(1-x) \dx + \int_1^c x(x-1) \dx, & c > 1,         \\[1ex]
          \int_0^c x(x-1) \dx,                       & c < 0,
        \end{cases} \\
        &= \begin{cases}
          c^2 \paren*{\frac12 - \frac{c}{3}},          & 0 \le c \le 1, \\[1ex]
          \frac16 + \frac{c^3-1}{3} - \frac{c^2-1}{2}, & c > 1          \\[1ex]
          c^2 \paren*{\frac{c}{3} - \frac12},          & c < 0,
        \end{cases} \\
        &= 0.
    \end{align*}
    The only solution for the above equation is \(c = 0\).

    Another way to solve the problem is to look at the geometrical
    interpretation of this integral.
  \end{enumerate}

\item Let \(f(x) = x(x^2-1)\), \(g(x) = x\), \(a = -1\), and \(b = \sqrt2\).
  Compute the area of the region \(S\) between the graphs of \(f\) and \(g\)
  over the interval \([a, b]\).  Make a sketch of the two graphs and
  indicate \(S\) by shading.

  We have
  \begin{align*}
    a(S) &= \int_{-1}^0 \paren*{\,f(x) - g(x)} \dx + \int_0^{\sqrt2} \paren*{g(x) - f(x)} \dx \\
         &= \int_{-1}^0 \paren[\big]{x^3 - 2x} \dx + \int_0^{\sqrt2} \paren[\big]{2x-x^3} \dx \\
         &= \int_0^1 \paren[\big]{2x - x^3} \dx + \int_0^{\sqrt2} \paren[\big]{2x-x^3} \dx \\
         &= 1 - \frac14 + 2 - \frac{2\sqrt2}{3} = \frac{11}{4} - \frac23\sqrt2.
  \end{align*}

  The sketch is as follows.

  \begin{figure}[H]
    \centering
    \tikzsetnextfilename{18.014-PS03.2}
    \begin{tikzpicture}[domain=-1.65:1.65]
      \draw[->] (-2,0) -- (2,0) node[anchor=west] {\(x\)};
      \draw[->] (0,-3) -- (0,3) node[anchor=south] {\(y\)};
      \draw[smooth] plot (\x,{\x*(\x*\x-1)}) node[right] {\(f(x) = x(x^2-1)\)};
      \draw plot (\x,\x) node[right] {\(g(x) = x\)};
      \fill[nearly transparent]
        (-1,-1) -- plot[domain=-1:0]    (\x,{\x*(\x*\x-1)}) -- cycle
                -- plot[domain=0:1.414] (\x,{\x*(\x*\x-1)}) -- cycle;
    \end{tikzpicture}
  \end{figure}

\item For step functions \(s\) and \(t\) defined on \([a, b]\), prove
  the Cauchy-Schwarz inequality
  \[
    \paren[\bigg]{\int_a^b s \cdot t}^2 \le \int_a^b s^2 \cdot \int_a^b t^2.
  \]
  Show that the equality if and only if \(s = ct\) for some
  \(c \in \mathbf{R}\).

\item[\bonus] Let
  \(B = \brce{x \in [0, 1] \mid x = m/2^n \text{ for some } m, n \in
    \mathbf{Z}}\).  Prove that the function
  \[
    f(x) =
    \begin{cases}
      1, & x \in B, \\
      0, & x \notin B,
    \end{cases}
  \]
  is not integrable on \([0, 1]\) by our definition of integrability.
\end{enumerate}
\end{document}
