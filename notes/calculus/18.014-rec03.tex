\documentclass[a4paper]{article}

\title{Solutions to Recitation~3}
\author{L. F. \textsc{Jaw}}

\usepackage[T1]{fontenc}
\usepackage{textcomp}
\usepackage{mathtools,amssymb,amsthm}
\usepackage[hmargin=1in,vmargin=1in]{geometry}
\usepackage{graphicx,cancel}
\usepackage{hyperref}
\hypersetup{%
  colorlinks=true,
  urlcolor=[rgb]{0,0.2,0.6},
  linkcolor={.},
  bookmarksdepth=2}
\usepackage{bookmark}

\frenchspacing

\newcommand*{\parasp}{\setlength{\parskip}{10pt plus 2pt minus 3pt}}
\newcommand*{\noparasp}{\setlength{\parskip}{0pt plus 1pt}}
\newcommand*{\setparasp}[1]{\setlength{\parskip}{#1}}
\newcommand*{\pskip}{\vskip 10pt plus 2pt minus 3pt}
% \newcommand\LEFTRIGHT[3]{\left#1 #3 \right#2}
\newcommand\SetSymbol[1][]{%
  \nonscript\:#1\vert
  \allowbreak
  \nonscript\:
  \mathopen{}}
% \newcommand*{\paren}[1]{\LEFTRIGHT(){#1}}
\DeclarePairedDelimiterX{\paren}[1]{\lparen}{\rparen}{%
  \renewcommand{\mid}{\SetSymbol[\delimsize]}#1}
% \newcommand*{\brkt}[1]{\LEFTRIGHT[]{#1}}
\DeclarePairedDelimiterX{\brkt}[1]{\lbrack}{\rbrack}{%
  \renewcommand{\mid}{\SetSymbol[\delimsize]}#1}
\DeclarePairedDelimiterX{\brce}[1]{\lbrace}{\rbrace}{%
  \renewcommand{\mid}{\SetSymbol[\delimsize]}#1}
\newcommand*{\unit}[1]{\,\mathrm{#1}}
\newcommand*{\DeclareUnit}[2]{\newcommand*{#1}{\unit{#2}}}
\DeclareUnit{\cm}{cm}
% \renewcommand*{\m}{\unit{m}}
\DeclareUnit{\m}{m}
\DeclareUnit{\kg}{kg}
\DeclareUnit{\s}{s}
\newcommand*{\R}{\mathbb{R}}
\newcommand*{\Z}{\mathbb{Z}}
% \newcommand*{\Rp}{(0,+\infty)}
% \newcommand*{\Rm}{(-\infty,0)}
\newcommand*{\deduce}{\mathrel{\Downarrow}}
% \newcommand*{\abs}[1]{\left\lvert #1 \right\rvert}
\DeclarePairedDelimiter{\abs}{\lvert}{\rvert}
% \newcommand*{\ceil}[1]{\left\lceil#1\right\rceil}
\DeclarePairedDelimiter{\ceil}{\lceil}{\rceil}
\newcommand*{\textop}[1]{\mathop{\text{#1}}}

\newcommand*{\enumparen}[1]{(\makebox[0.6em][c]{#1})}
\renewcommand{\labelenumii}{\enumparen{\theenumii}}

\DeclareMathOperator{\arccosh}{arccosh}
\DeclareMathOperator{\gammaf}{\Gamma}
\DeclareMathOperator{\var}{var}
\DeclareMathOperator{\Ber}{Bernoulli}
\DeclareMathOperator{\Cov}{Cov}
\DeclareMathOperator{\E}{E}
\def\argmax{\qopname\relax m{arg\,max}}
\DeclarePairedDelimiterXPP{\Eb}[1]{\E}{\lbrack}{\rbrack}{}{%
  \renewcommand{\mid}{\SetSymbol[\delimsize]}#1}
\DeclarePairedDelimiterXPP{\varp}[1]{\var}{\lparen}{\rparen}{}{%
  \renewcommand{\mid}{\SetSymbol[\delimsize]}#1}
\DeclarePairedDelimiterXPP{\Covp}[1]{\Cov}{\lparen}{\rparen}{}{%
  \renewcommand{\mid}{\SetSymbol[\delimsize]}#1}
\DeclarePairedDelimiterXPP{\expp}[1]{\exp}{\lbrace}{\rbrace}{}{#1}
\renewcommand*{\Pr}{\mathop{P}}
\DeclarePairedDelimiterXPP{\Prp}[1]{\Pr}{\lparen}{\rparen}{}{%
  \renewcommand{\mid}{\SetSymbol[\delimsize]}#1}
\newcommand*{\pnorm}{\mathop{\Phi}}
\DeclarePairedDelimiterXPP{\pnormp}[1]{\pnorm}{\lparen}{\rparen}{}{#1}
\newcommand*{\dnorm}{\mathop{\varphi}}
\DeclarePairedDelimiterXPP{\dnormp}[1]{\dnorm}{\lparen}{\rparen}{}{#1}
\newcommand*{\qnorm}{\mathop{\Phi^{-1}}}
%\newcommand*{\diff}{\mathop{}\!d}
\newcommand*{\diff}{\mathop{}\!\mathit{d}}
%\newcommand*{\diff}{\mathop{}\!\mathrm{d}}
\newcommand*{\dx}{\diff x}
\newcommand*{\dy}{\diff y}
\newcommand*{\dz}{\diff z}
\newcommand*{\ds}{\diff s}
\newcommand*{\dt}{\diff t}
\newcommand*{\du}{\diff u}
\newcommand*{\dv}{\diff v}
\newcommand*{\dtheta}{\diff \theta}
\newcommand*{\dd}[2][]{\frac{\diff#1}{\diff#2}}
\newcommand*{\ddx}{\frac{\diff}{\dx}}
\newcommand*{\ddt}{\frac{\diff}{\dt}}
\newcommand*{\ddy}{\dd y}
\newcommand*{\ddtheta}{\frac{\diff}{\dtheta}}
\newcommand*{\ddz}{\dd z}
\newcommand*{\fwdf}{\mathop{}\!\Delta}
\newcommand*{\dydx}{\frac\dy\dx}
\newcommand*{\pdpd}[2][]{\frac{\partial#1}{\partial#2}}
\newcommand*{\pdpdx}{\frac\partial{\partial x}}
\newcommand*{\pdpdy}{\frac\partial{\partial y}}
\newcommand*{\pdpdz}{\frac\partial{\partial z}}
\newcommand*{\pdpdu}{\frac\partial{\partial u}}
\newcommand*{\pdpdv}{\frac\partial{\partial v}}
\newcommand*{\pdpdt}{\frac\partial{\partial t}}
\newcommand*{\pdzpdx}{\frac{\partial z}{\partial x}}
\newcommand*{\pdzpdy}{\frac{\partial z}{\partial y}}
\newcommand*{\pdzpdt}{\frac{\partial z}{\partial t}}
\newcommand*{\pdxpdt}{\frac{\partial x}{\partial t}}
\newcommand*{\pdypdt}{\frac{\partial y}{\partial t}}

% \usepackage[lite,subscriptcorrection,nofontinfo]{mtpro2}
\usepackage{fontspec}

\setmainfont{Palatino Linotype}[Ligatures=TeX,Numbers=OldStyle]
\setmonofont{Source Code Pro}
% \usepackage[integrals]{wasysym}
\usepackage{fontawesome}

\usepackage[math-style=TeX]{unicode-math}
\setmathfont{TeX Gyre Pagella Math}

\usepackage{microtype}

\let\reason\text
\let\vect\symbf

\AtBeginDocument{%
  % \renewcommand{\perp}{\mathrel{\bot}}
  \let\leq\leqslant
  \let\le\leq
  \let\geq\geqslant
  \let\ge\geq}


\begin{document}
\maketitle

This is my own solution to the Recitation~3 of
\href{https://ocw.mit.edu/courses/mathematics/18-014-calculus-with-theory-fall-2010/recitations/}{18.014},
which includes Theorem~I.24 and Exercise~1a, 11, and 6 from Apostol's
\textit{Calculus} (1: 20, 35--36, 40).

\begin{enumerate}
\item If \(ab > 0\), then both \(a\) and \(b\) are positive or both
  negative.

  \begin{proof}
    If either \(a\) or \(b\) is zero, then \(ab = 0\) by Theorem~I.6.  But
    by Axiom~9, \(0 \ngtr 0\).  Thus, neither \(a\) nor \(b\) is zero.  If
    one is negative and the other positive, say, \(a < 0\) and \(b > 0\),
    then \(-a > 0\) by Theorem~I.23 and \(-ab = (-a)b > 0\) by Theorem~I.12
    and Axiom~7.  However, this contradicts with Axiom~8 since both
    \(ab > 0\) and \(-ab >0\).  It is easy to verify that no contradiction
    will be derived in the remaining scenarios.  % If both are negative, then
    % \(-a > 0\) and \(-b > 0\) by Theorem~I.23, and \(ab = (-a)(-b) > 0\) by
    % Theorem~I.12 and Axiom~7.  If both are positive, then it follows that
    % \(ab > 0\) by Axiom~7 solely.
  \end{proof}
  
\item Prove \(1 + 2 + 3 + \dotsb + n = n(n+1)/2 \) by induction.

  \begin{proof}
    The basis is clearly true for \(n = 1\) since \(1 = 1 \cdot (1+1)/2\).
    Now suppose the above holds for some \(k \ge 1\).  Then
    \begin{align*}
      1 + 2 + 3 + \dotsb + k + (k+1)
        &= \frac{k(k+1)}{2} + (k+1) \\
        &= \paren*{\frac{k}{2} + 1} (k+1) \\
        &= \frac{(k+2)(k+1)}{2} \\
        &= \frac{(k+1)\brkt*{(k+1)+1}}{2}. \qedhere
    \end{align*}
  \end{proof}
  
\item Let \(n\) and \(d\) denote integers.  We say that \(d\) is a
  \emph{divisor} of \(n\) if \(n = cd\) for some integer \(c\).  An
  integer \(n\) is called a \emph{prime} if \(n > 1\) and if the only
  positive divisors of \(n\) are \(1\) and \(n\).  Prove, by induction,
  that every integer \(n > 1\) is either a prime or a product of primes.

  \begin{proof}
    It's easy to verify that \(2\) is a prime and its only positive
    divisors are \(1\) and \(2\).  Now suppose the above statement holds
    for some \(n > 1\).  We are going to establish that \(n+1\) is either
    a prime or a product of primes.  Suppose \(n+1\) is not a prime.  Then
    by definition and simple reasoning there exist integers \(c\) and
    \(d\) such that \(1 < c, d < n+1\).  Then by the induction hypothesis,
    \(c\) and \(d\) are either a prime or a product of primes, which
    implies \(n+1\) is a product of primes.  Thus, \(n+1\) is either a
    prime or a product of primes (constructive dilemma).
  \end{proof}
  
\item Derive the formula
  \begin{displaymath}
    \sum_{k=1}^n k^2 = \frac{n^2}{3} + \frac{n^2}{2} + \frac{n}{6}.
  \end{displaymath}

  We have
  \begin{align*}
    3 \paren*{\sum_{k=1}^n k^2} - 3 \paren*{\sum_{k=1}^n k}  + n
      &= \sum_{k=1}^n (3k^2 - 3k + 1) && \reason{by linearity,} \\
      &= \sum_{k=1}^n \brkt[\big]{k^3 - \paren{k-1}^3} \\
      &= n^3 - 0^3 && \reason{by telescoping property,} \\
      &= n^3.
  \end{align*}
  We already know that \(\sum_{k=1}^n k = n^2/2 + n/2\).  Substitute this
  into the above equation, do some simple algebra, and we obtain
  \begin{displaymath}
    \sum_{k=1}^n k^2
      = \frac13 \paren*{n^3 + \frac32 n^2 + \frac32 n - n}
      = \frac{n^3}{3} + \frac{n^2}{2} + \frac{n}{6}.
  \end{displaymath}
\end{enumerate}
\end{document}
