\documentclass[a4paper]{article}

\title{Solutions to Recitation 5}
\author{L. F. \textsc{Jaw}}

\usepackage[T1]{fontenc}
\usepackage{textcomp}
\usepackage{mathtools,amssymb,amsthm}
\usepackage[hmargin=1in,vmargin=1in]{geometry}
\usepackage{graphicx,cancel}
\usepackage{hyperref}
\hypersetup{%
  colorlinks=true,
  urlcolor=[rgb]{0,0.2,0.6},
  linkcolor={.},
  bookmarksdepth=2}
\usepackage{bookmark}

\frenchspacing

\newcommand*{\parasp}{\setlength{\parskip}{10pt plus 2pt minus 3pt}}
\newcommand*{\noparasp}{\setlength{\parskip}{0pt plus 1pt}}
\newcommand*{\setparasp}[1]{\setlength{\parskip}{#1}}
\newcommand*{\pskip}{\vskip 10pt plus 2pt minus 3pt}
% \newcommand\LEFTRIGHT[3]{\left#1 #3 \right#2}
\newcommand\SetSymbol[1][]{%
  \nonscript\:#1\vert
  \allowbreak
  \nonscript\:
  \mathopen{}}
% \newcommand*{\paren}[1]{\LEFTRIGHT(){#1}}
\DeclarePairedDelimiterX{\paren}[1]{\lparen}{\rparen}{%
  \renewcommand{\mid}{\SetSymbol[\delimsize]}#1}
% \newcommand*{\brkt}[1]{\LEFTRIGHT[]{#1}}
\DeclarePairedDelimiterX{\brkt}[1]{\lbrack}{\rbrack}{%
  \renewcommand{\mid}{\SetSymbol[\delimsize]}#1}
\DeclarePairedDelimiterX{\brce}[1]{\lbrace}{\rbrace}{%
  \renewcommand{\mid}{\SetSymbol[\delimsize]}#1}
\newcommand*{\unit}[1]{\,\mathrm{#1}}
\newcommand*{\DeclareUnit}[2]{\newcommand*{#1}{\unit{#2}}}
\DeclareUnit{\cm}{cm}
% \renewcommand*{\m}{\unit{m}}
\DeclareUnit{\m}{m}
\DeclareUnit{\kg}{kg}
\DeclareUnit{\s}{s}
\newcommand*{\R}{\mathbb{R}}
\newcommand*{\Z}{\mathbb{Z}}
% \newcommand*{\Rp}{(0,+\infty)}
% \newcommand*{\Rm}{(-\infty,0)}
\newcommand*{\deduce}{\mathrel{\Downarrow}}
% \newcommand*{\abs}[1]{\left\lvert #1 \right\rvert}
\DeclarePairedDelimiter{\abs}{\lvert}{\rvert}
% \newcommand*{\ceil}[1]{\left\lceil#1\right\rceil}
\DeclarePairedDelimiter{\ceil}{\lceil}{\rceil}
\newcommand*{\textop}[1]{\mathop{\text{#1}}}

\newcommand*{\enumparen}[1]{(\makebox[0.6em][c]{#1})}
\renewcommand{\labelenumii}{\enumparen{\theenumii}}

\DeclareMathOperator{\arccosh}{arccosh}
\DeclareMathOperator{\gammaf}{\Gamma}
\DeclareMathOperator{\var}{var}
\DeclareMathOperator{\Ber}{Bernoulli}
\DeclareMathOperator{\Cov}{Cov}
\DeclareMathOperator{\E}{E}
\def\argmax{\qopname\relax m{arg\,max}}
\DeclarePairedDelimiterXPP{\Eb}[1]{\E}{\lbrack}{\rbrack}{}{%
  \renewcommand{\mid}{\SetSymbol[\delimsize]}#1}
\DeclarePairedDelimiterXPP{\varp}[1]{\var}{\lparen}{\rparen}{}{%
  \renewcommand{\mid}{\SetSymbol[\delimsize]}#1}
\DeclarePairedDelimiterXPP{\Covp}[1]{\Cov}{\lparen}{\rparen}{}{%
  \renewcommand{\mid}{\SetSymbol[\delimsize]}#1}
\DeclarePairedDelimiterXPP{\expp}[1]{\exp}{\lbrace}{\rbrace}{}{#1}
\renewcommand*{\Pr}{\mathop{P}}
\DeclarePairedDelimiterXPP{\Prp}[1]{\Pr}{\lparen}{\rparen}{}{%
  \renewcommand{\mid}{\SetSymbol[\delimsize]}#1}
\newcommand*{\pnorm}{\mathop{\Phi}}
\DeclarePairedDelimiterXPP{\pnormp}[1]{\pnorm}{\lparen}{\rparen}{}{#1}
\newcommand*{\dnorm}{\mathop{\varphi}}
\DeclarePairedDelimiterXPP{\dnormp}[1]{\dnorm}{\lparen}{\rparen}{}{#1}
\newcommand*{\qnorm}{\mathop{\Phi^{-1}}}
%\newcommand*{\diff}{\mathop{}\!d}
\newcommand*{\diff}{\mathop{}\!\mathit{d}}
%\newcommand*{\diff}{\mathop{}\!\mathrm{d}}
\newcommand*{\dx}{\diff x}
\newcommand*{\dy}{\diff y}
\newcommand*{\dz}{\diff z}
\newcommand*{\ds}{\diff s}
\newcommand*{\dt}{\diff t}
\newcommand*{\du}{\diff u}
\newcommand*{\dv}{\diff v}
\newcommand*{\dtheta}{\diff \theta}
\newcommand*{\dd}[2][]{\frac{\diff#1}{\diff#2}}
\newcommand*{\ddx}{\frac{\diff}{\dx}}
\newcommand*{\ddt}{\frac{\diff}{\dt}}
\newcommand*{\ddy}{\dd y}
\newcommand*{\ddtheta}{\frac{\diff}{\dtheta}}
\newcommand*{\ddz}{\dd z}
\newcommand*{\fwdf}{\mathop{}\!\Delta}
\newcommand*{\dydx}{\frac\dy\dx}
\newcommand*{\pdpd}[2][]{\frac{\partial#1}{\partial#2}}
\newcommand*{\pdpdx}{\frac\partial{\partial x}}
\newcommand*{\pdpdy}{\frac\partial{\partial y}}
\newcommand*{\pdpdz}{\frac\partial{\partial z}}
\newcommand*{\pdpdu}{\frac\partial{\partial u}}
\newcommand*{\pdpdv}{\frac\partial{\partial v}}
\newcommand*{\pdpdt}{\frac\partial{\partial t}}
\newcommand*{\pdzpdx}{\frac{\partial z}{\partial x}}
\newcommand*{\pdzpdy}{\frac{\partial z}{\partial y}}
\newcommand*{\pdzpdt}{\frac{\partial z}{\partial t}}
\newcommand*{\pdxpdt}{\frac{\partial x}{\partial t}}
\newcommand*{\pdypdt}{\frac{\partial y}{\partial t}}

% \usepackage[lite,subscriptcorrection,nofontinfo]{mtpro2}
\usepackage{fontspec}

\setmainfont{Palatino Linotype}[Ligatures=TeX,Numbers=OldStyle]
\setmonofont{Source Code Pro}
% \usepackage[integrals]{wasysym}
\usepackage{fontawesome}

\usepackage[math-style=TeX]{unicode-math}
\setmathfont{TeX Gyre Pagella Math}

\usepackage{microtype}

\let\reason\text
\let\vect\symbf

\AtBeginDocument{%
  % \renewcommand{\perp}{\mathrel{\bot}}
  \let\leq\leqslant
  \let\le\leq
  \let\geq\geqslant
  \let\ge\geq}


\begin{document}
\maketitle

This is my own solution to the Recitation~5 of
\href{https://ocw.mit.edu/courses/mathematics/18-014-calculus-with-theory-fall-2010/recitations/}{18.014},
which includes Exercise~9abc, 11, 1ab, and 4a from Apostol's
\textit{Calculus} (1: 57, 63).

\begin{enumerate}
\item This exercise develops some fundamental properties of
  polynomials.  Let \(f(x) = \sum_{k=0}^n c_k x^k\) be a polynomial of
  degree \(n\).  Prove each of the following:
  \begin{enumerate}
  \item If \(n \ge 1\) and \(f(0) = 0\), then \(f(x) = x \, g(x)\), where
    \(g\) is a polynomial of degree \(n-1\).

    \begin{proof}
      Since \(n \ge 1\), we can rewrite \(f\) as
      \begin{displaymath}
        f(x) = c_0 + \sum_{k=1}^n c_k x^k = c_0 + x \sum_{k=1}^n c_k x^{k-1}.
      \end{displaymath}
      But \(f(0) = 0\), which implies \(c_0 = 0\).  Then let
      \(g = \sum_{k=1}^n c_k x^{k-1}\) and we obtain
      \begin{displaymath}
        f(x) = x \sum_{k=1}^n c_k x^{k-1} = x \, g(x).
      \end{displaymath}
    \end{proof}

  \item For each real \(a\), the function \(p\) given by
    \(p(x) = f(x+a)\) is a polynomial of degree \(n\).

    \begin{proof}
      We have
      \begin{align*}
        f(x+a) &= \sum_{k=0}^n c_k \paren{x+a}^k \\
               &= \sum_{k=0}^n \brce*{ c_k \sum_{i=0}^k \binom{k}{i} a^{k-i} x^i } \\
               &= \sum_{k=0}^n \sum_{i=0}^k \binom{k}{i} c_k a^{k-i} x^i \\
               &= \sum_{k=0}^n \brce*{ \sum_{i=0}^{n-k} \binom{k+i}{k} c_{k+i} a^{i} } x^k. \qedhere
      \end{align*}
    \end{proof}
    The above proof says too much, which not only shows the new
    polynomial is of the same degree but also provide a way to compute
    the coefficients for it.  A less combinatorial proof is as
    follows.

    \begin{proof}
      The statement clearly holds for \(n = 0\) since \(f(x+a) = f(x) = c_0\).
      Suppose the statement holds for some \(n \ge 0\) and \(f\) is a polynomial
      of degree \(n+1\).  We decompose \(f(x+a)\) into
      \begin{displaymath}
        f(x+a) = \sum_{k=0}^{n+1} c_k \paren{x+a}^k = \paren*{ \sum_{k=0}^n c_k \paren{x+a}^k } + c_{n+1} \paren{x+a}^{n+1}.
      \end{displaymath}
      Let \(g(x) = \sum_{k=0}^n c_k x^k\).  Since \(g\) is a polynomial of
      degree \(n\), then there must be a sequence of coefficients \(d_k\) such
      that \(g(x+a) = \sum_{k=0}^n d_k x^k\).  Thus
      \begin{align*}
        f(x+a) &= g(x+a) + c_{n+1} \paren{x+a}^{n+1} \\
               &= \sum_{k=0}^n d_k x^k + c_{n+1} \sum_{k=0}^{n+1} \binom{n+1}{k} a^{n+1-k} x^k \\
               &= \paren*{ \sum_{k=0}^n d_k x^k } + \brce*{ \sum_{k=0}^n \binom{n+1}{k} c_{n+1} a^{n+1-k}x^k } + c_{n+1} x^{n+1} \\
               &= \brce*{ \sum_{k=0}^n \brkt*{ d_k + \binom{n+1}{k} c_{n+1} a^{n+1-k} } x^k } + c_{n+1} x^{n+1}.
      \end{align*}
      Let
      \begin{displaymath}
        e_k =
        \begin{cases}
          d_k + \binom{n+1}{k} c_{n+1} a^{n+1-k}, &  0 \le k \le n, \\
          c_{n+1}, & k = n+1.
        \end{cases}
      \end{displaymath}
      Then we could rewrite \(f(x+a)\) as
      \begin{displaymath}
        f(x+a) = \paren*{ \sum_{k=0}^n e_k x^k } + e_{n+1} x^{n+1} = \sum_{k=0}^{n+1} e_k x^k. \qedhere
      \end{displaymath}
    \end{proof}

  \item If \(n \ge 1\) and \(f(a) = 0\) for some real \(a\), then
    \(f(x) = (x-a) \, h(x)\), where \(h\) is a polynomial of degree
    \(n-1\).

    \begin{proof}
      Let \(g(x) = f(x+a)\).  By the result from \enumparen{b}, \(g\) is a
      polynomial of degree \(n\).  Since \(n \ge 1\) and
      \(g(0) = f(a) = 0\), by the result from \enumparen{a},
      \(g(x) = x \, p(x) \), where \(p(x)\) is a polynomial of degree
      \(n-1\).  Let \(h(x) = p(x-a)\).  Again by the result from
      \enumparen{b}, \(h\) is also a polynomial of degree \(n-1\).  Then
      \(f(x) = g(x-a) = (x-a) \, p(x-a) = (x-a) \, h(x)\).
    \end{proof}
  \end{enumerate}

\item In each case, find all polynomials \(p\) of degree \(\le 2\) which
  satisfy the given conditions for all real \(x\).
  \begin{enumerate}
  \item \(p(x) = p(1-x)\).

    The coefficient \(c_0\) can be any real number regardless of the
    degree.

    When the degree is zero, the condition clearly holds for all constant
    polynonimals.

    When the degree is \(1\) (\(c_1 \ne 0\)), we have
    \begin{gather*}
      c_1 x = c_1 (1-x), \text{ or} \\
      c_1 (2x-1) = 0.
    \end{gather*}
    Since \(c_1\) is not zero, it must be the case that \(x = 1/2\).  This
    means there is no polynonimal of degree \(1\) which satisfies the
    above condition for all real \(x\).

    When the degree is \(2\) (\(c_2 \ne 0\)), we have
    \begin{displaymath}
      c_1 x + c_2 x^2 = c_1 (1-x) + c_2 \paren{1-x}^2,
    \end{displaymath}
    which simplies to
    \begin{displaymath}
      (c_1 + c_2)(2x - 1) = 0.
    \end{displaymath}
    By the same reasoning in the case of degree \(1\), it must be the case
    that \(c_1 + c_2 = 0\).

  \item \(p(x) = p(1+x)\).

    As in the previous exercise, the coefficient \(c_0\) can be any real
    number regardless of the degree.

    By the same reasoning, it's easy to show that there is no polynonimal
    of degree \(1\) which satisfies the above condition for all real
    \(x\).

    When the degree is \(2\) (\(c_2 \ne 0\)), we have
    \begin{displaymath}
      c_1 x + c_2 x^2 = c_1 (1+x) + c_2 \paren{1+x}^2,
    \end{displaymath}
    which simplies to
    \begin{displaymath}
      c_1 + c_2 (2x+1) = 0.
    \end{displaymath}
    Since \(c_2\) is not zero, no matter what value \(c_1\) is, we can
    always find a real number \(x\) such that \(c_1 + c_2 (2x+1) \ne 0\).
    This means there is no polynonimal of degree \(2\) which satisfies the
    above condition for all real \(x\) either.

  \item \(p(2x) = 2p(x)\).

    The coefficient \(c_0\) must be zero regardless of the degree for the
    above condition to hold for all real \(x\), since
    \begin{gather*}
      p(2\cdot 0) = 2p(0), \text{ or} \\
      p(0) = c_0 = 0.
    \end{gather*}

    When the degree is zero, the only polynonimal satisfying the stated
    condition for all real \(x\) is \(p(x) = 0\).

    When the degree is \(1\), we have
    \begin{displaymath}
      c_0 + c_1 (2x) = 2(c_0 + c_1 x)
    \end{displaymath}
    and since \(c_0 = 0\), the above equality can be simplified into
    \begin{displaymath}
      2 c_1 x = 2 c_1 x,
    \end{displaymath}
    which is a tautology.  This means \(c_1\) can take any value except
    zero as long as \(c_0\) is zero.

    When the degree is \(2\), we have
    \begin{gather*}
      c_2 \paren{2x}^2 = 2 c_2 x^2, \text{ or} \\
      2c_2x^2 = 0.
    \end{gather*}
    Since \(c_2 \ne 0\), the last equality is only satisfied when
    \(x = 0\).  This means there is no polynonimal of degree \(2\)
    satisfying the stated condition for all real \(x\).

  \item \(p(3x) = p(x+3)\).

    As before, the coefficient \(c_0\) can be any real number regardless
    of the degree.

    When the degree is zero, the stated condition holds for all constant
    polynonimals.

    When the degree is \(1\) (\(c_1 \ne 0\)), we have
    \begin{gather*}
      c_1 (3x) = c_1 (x+3), \text{ or} \\
      x = \frac32.
    \end{gather*}
    This means there is no polynonimal of degree \(1\) satisfying the
    stated condition for all real \(x\).

    When the degree is \(2\), we have
    \begin{gather*}
      c_1 (3x) + c_2 \paren{3x}^2 = c_1 (x+3) + c_2 \paren{x+3}^2, \text{ or} \\
      \brkt*{c_1 + c_2 (4x+3)} (2x-3) = 0.
    \end{gather*}
    Solve the last equation and we obtain
    \begin{displaymath}
      x = -\frac14 \paren*{\frac{c_1}{c_2} + 3} \quad \text{or} \quad x = \frac32.
    \end{displaymath}
    This means there is no polynonimal of degree \(2\) satisfying the
    stated condition for all real \(x\).
  \end{enumerate}
\end{enumerate}
\end{document}
