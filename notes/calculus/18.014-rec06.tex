\documentclass[a4paper]{article}

\title{Solutions to Recitation~6}
\author{L. F. \textsc{Jaw}}

\usepackage[T1]{fontenc}
\usepackage{textcomp}
\usepackage{mathtools,amssymb,amsthm}
\usepackage[hmargin=1in,vmargin=1in]{geometry}
\usepackage{graphicx,cancel}
\usepackage{hyperref}
\hypersetup{%
  colorlinks=true,
  urlcolor=[rgb]{0,0.2,0.6},
  linkcolor={.},
  bookmarksdepth=2}
\usepackage{bookmark}

\frenchspacing

\newcommand*{\parasp}{\setlength{\parskip}{10pt plus 2pt minus 3pt}}
\newcommand*{\noparasp}{\setlength{\parskip}{0pt plus 1pt}}
\newcommand*{\setparasp}[1]{\setlength{\parskip}{#1}}
\newcommand*{\pskip}{\vskip 10pt plus 2pt minus 3pt}
% \newcommand\LEFTRIGHT[3]{\left#1 #3 \right#2}
\newcommand\SetSymbol[1][]{%
  \nonscript\:#1\vert
  \allowbreak
  \nonscript\:
  \mathopen{}}
% \newcommand*{\paren}[1]{\LEFTRIGHT(){#1}}
\DeclarePairedDelimiterX{\paren}[1]{\lparen}{\rparen}{%
  \renewcommand{\mid}{\SetSymbol[\delimsize]}#1}
% \newcommand*{\brkt}[1]{\LEFTRIGHT[]{#1}}
\DeclarePairedDelimiterX{\brkt}[1]{\lbrack}{\rbrack}{%
  \renewcommand{\mid}{\SetSymbol[\delimsize]}#1}
\DeclarePairedDelimiterX{\brce}[1]{\lbrace}{\rbrace}{%
  \renewcommand{\mid}{\SetSymbol[\delimsize]}#1}
\newcommand*{\unit}[1]{\,\mathrm{#1}}
\newcommand*{\DeclareUnit}[2]{\newcommand*{#1}{\unit{#2}}}
\DeclareUnit{\cm}{cm}
% \renewcommand*{\m}{\unit{m}}
\DeclareUnit{\m}{m}
\DeclareUnit{\kg}{kg}
\DeclareUnit{\s}{s}
\newcommand*{\R}{\mathbb{R}}
\newcommand*{\Z}{\mathbb{Z}}
% \newcommand*{\Rp}{(0,+\infty)}
% \newcommand*{\Rm}{(-\infty,0)}
\newcommand*{\deduce}{\mathrel{\Downarrow}}
% \newcommand*{\abs}[1]{\left\lvert #1 \right\rvert}
\DeclarePairedDelimiter{\abs}{\lvert}{\rvert}
% \newcommand*{\ceil}[1]{\left\lceil#1\right\rceil}
\DeclarePairedDelimiter{\ceil}{\lceil}{\rceil}
\newcommand*{\textop}[1]{\mathop{\text{#1}}}

\newcommand*{\enumparen}[1]{(\makebox[0.6em][c]{#1})}
\renewcommand{\labelenumii}{\enumparen{\theenumii}}

\DeclareMathOperator{\arccosh}{arccosh}
\DeclareMathOperator{\gammaf}{\Gamma}
\DeclareMathOperator{\var}{var}
\DeclareMathOperator{\Ber}{Bernoulli}
\DeclareMathOperator{\Cov}{Cov}
\DeclareMathOperator{\E}{E}
\def\argmax{\qopname\relax m{arg\,max}}
\DeclarePairedDelimiterXPP{\Eb}[1]{\E}{\lbrack}{\rbrack}{}{%
  \renewcommand{\mid}{\SetSymbol[\delimsize]}#1}
\DeclarePairedDelimiterXPP{\varp}[1]{\var}{\lparen}{\rparen}{}{%
  \renewcommand{\mid}{\SetSymbol[\delimsize]}#1}
\DeclarePairedDelimiterXPP{\Covp}[1]{\Cov}{\lparen}{\rparen}{}{%
  \renewcommand{\mid}{\SetSymbol[\delimsize]}#1}
\DeclarePairedDelimiterXPP{\expp}[1]{\exp}{\lbrace}{\rbrace}{}{#1}
\renewcommand*{\Pr}{\mathop{P}}
\DeclarePairedDelimiterXPP{\Prp}[1]{\Pr}{\lparen}{\rparen}{}{%
  \renewcommand{\mid}{\SetSymbol[\delimsize]}#1}
\newcommand*{\pnorm}{\mathop{\Phi}}
\DeclarePairedDelimiterXPP{\pnormp}[1]{\pnorm}{\lparen}{\rparen}{}{#1}
\newcommand*{\dnorm}{\mathop{\varphi}}
\DeclarePairedDelimiterXPP{\dnormp}[1]{\dnorm}{\lparen}{\rparen}{}{#1}
\newcommand*{\qnorm}{\mathop{\Phi^{-1}}}
%\newcommand*{\diff}{\mathop{}\!d}
\newcommand*{\diff}{\mathop{}\!\mathit{d}}
%\newcommand*{\diff}{\mathop{}\!\mathrm{d}}
\newcommand*{\dx}{\diff x}
\newcommand*{\dy}{\diff y}
\newcommand*{\dz}{\diff z}
\newcommand*{\ds}{\diff s}
\newcommand*{\dt}{\diff t}
\newcommand*{\du}{\diff u}
\newcommand*{\dv}{\diff v}
\newcommand*{\dtheta}{\diff \theta}
\newcommand*{\dd}[2][]{\frac{\diff#1}{\diff#2}}
\newcommand*{\ddx}{\frac{\diff}{\dx}}
\newcommand*{\ddt}{\frac{\diff}{\dt}}
\newcommand*{\ddy}{\dd y}
\newcommand*{\ddtheta}{\frac{\diff}{\dtheta}}
\newcommand*{\ddz}{\dd z}
\newcommand*{\fwdf}{\mathop{}\!\Delta}
\newcommand*{\dydx}{\frac\dy\dx}
\newcommand*{\pdpd}[2][]{\frac{\partial#1}{\partial#2}}
\newcommand*{\pdpdx}{\frac\partial{\partial x}}
\newcommand*{\pdpdy}{\frac\partial{\partial y}}
\newcommand*{\pdpdz}{\frac\partial{\partial z}}
\newcommand*{\pdpdu}{\frac\partial{\partial u}}
\newcommand*{\pdpdv}{\frac\partial{\partial v}}
\newcommand*{\pdpdt}{\frac\partial{\partial t}}
\newcommand*{\pdzpdx}{\frac{\partial z}{\partial x}}
\newcommand*{\pdzpdy}{\frac{\partial z}{\partial y}}
\newcommand*{\pdzpdt}{\frac{\partial z}{\partial t}}
\newcommand*{\pdxpdt}{\frac{\partial x}{\partial t}}
\newcommand*{\pdypdt}{\frac{\partial y}{\partial t}}

% \usepackage[lite,subscriptcorrection,nofontinfo]{mtpro2}
\usepackage{fontspec}

\setmainfont{Palatino Linotype}[Ligatures=TeX,Numbers=OldStyle]
\setmonofont{Source Code Pro}
% \usepackage[integrals]{wasysym}
\usepackage{fontawesome}

\usepackage[math-style=TeX]{unicode-math}
\setmathfont{TeX Gyre Pagella Math}

\usepackage{microtype}

\let\reason\text
\let\vect\symbf

\AtBeginDocument{%
  % \renewcommand{\perp}{\mathrel{\bot}}
  \let\leq\leqslant
  \let\le\leq
  \let\geq\geqslant
  \let\ge\geq}


\begin{document}
\maketitle

This is my own solution to the Recitation~6 from
\href{https://ocw.mit.edu/courses/mathematics/18-014-calculus-with-theory-fall-2010/recitations/}{18.014},
which includes Exercise~1bcde, 2, 3, and 13 from Apostol's
\textit{Calculus} (1: 70--71).

\begin{enumerate}
\item Compute the value of each of the following integrals.  You may use
  theorems of Section 1.13 whenever it is convenient to do so.  The
  notation \([x]\) denotes the greatest integer \(\le x\).
  \begin{enumerate}
    \everymath{\displaystyle}
  \item \(\int_{-1}^3 [x + \tfrac12] \dx\).

    \begin{displaymath}
      \int_{-1}^3 [x + \tfrac12] \dx
      = \int_{-0.5}^{3.5} [x] \dx
      = -1 \times 0.5 + 0 + 1 + 2 + 3 \times 0.5
      = 4.
    \end{displaymath}

  \item \(\int_{-1}^3 \paren[\big]{[x] + [x + \tfrac12]} \dx\).

    \begin{displaymath}
      \int_{-1}^3 \paren[\big]{[x] + [x + \tfrac12]} \dx
        = \int_{-1}^3 [x] \dx + \int_{-1}^3 [x + \tfrac12] \dx
        = 2 + 4 = 6.
    \end{displaymath}

  \item \(\int_{-1}^3 2[x] \dx\).

    \begin{displaymath}
      \int_{-1}^3 2[x] \dx = 2 \int_{-1}^3 [x] \dx = 2 \times 2 = 4.
    \end{displaymath}

  \item \(\int_{-1}^3 [2x] \dx\).

    \begin{displaymath}
      \int_{-1}^3 [2x] \dx
        = \frac12 \int_{-2}^6 [x] \dx
        = \frac12 \paren*{\int_{-2}^{-1} [x] \dx + \int_{-1}^3 [x] \dx
          + \int_3^6 [x] \dx}
        = \frac12 (-2 + 2 + 3 + 4 + 5)
        = 6.
    \end{displaymath}
  \end{enumerate}

\item Give an example of a step function \(s\), defined on the closed
  interval \([0, 5]\), which has the following properties:
  \[
    \int_0^2 s(x) \dx = 5
    \quad \text{and} \quad
    \int_0^5 s(x) \dx = 2.
  \]

  Simply define
  \[
    s(x) = \begin{cases}
      \frac52, & 0 \le x < 2, \\[1ex]
      -1,      & 2 \le x \le 5.
    \end{cases}
  \]

\item Show that \(\int_a^b [x] \dx + \int_a^b [-x] \dx = a-b\).

  \begin{proof}
    By linearity, we have
    \[
      \int_a^b [x] \dx + \int_a^b [-x] \dx
        = \int_a^b \paren[\big]{[x] + [-x]} \dx.
    \]
    Let us define \(f(x) = [x] + [-x]\).  Notice that
    \[
      f(x) =
      \begin{cases}
        0, & x \text{ is an integer,} \\
        -1, & \text{otherwise.}
      \end{cases}
    \]

    The case for \(a = b\) is trivial.  When \(a < b\), we can find a
    partitation \(P = \{x_0, x_1, \dotsc, x_n\}\), where \(x_0 = a\),
    \(x_n = b\), and \(x_k \in \mathbf{Z}\) for all \(a < k < b\).  Then
    we have
    \begin{align*}
      \int_a^b f(x) \dx
        &= \sum_{k=1}^n (-1)(x_k - x_{k-1})  && \reason{by definition of integral of step functions,} \\
        &= (-1) \sum_{k=1}^n (x_k - x_{k-1}) && \reason{by homogeneity of summation,} \\
        &= (-1) (x_n - x_{0})                && \reason{by telescoping property of summation,} \\
        &= a - b                             && \reason{by our choice of partition and simple algebra.}
    \end{align*}
    % It is easy to verify that \(f\) is an even function.  When \(a > b\),
    % we have
    % \begin{align*}
    %   \int_a^b f(x) \dx
    %     &= \int_{-b}^{-a} f(-x) \dx && \reason{by reflection property,} \\
    %     &= \int_{-b}^{-a} f(x) \dx  && \reason{by definition of even function,} \\
    %     &= -b - (-a)                && \reason{by the fact that \(-b < -a\),} \\
    %     &= a - b                    && \reason{by simple algebra.} \qedhere
    % \end{align*}
    When \(a > b\), we have
    \begin{align*}
      \int_a^b f(x) \dx
        &= -\int_b^a f(x) \dx && \reason{by our convention,} \\
        &= -(b-a)             && \reason{by the fact that \(b < a\),} \\
        &= a - b              && \reason{by simple algebra.} \qedhere
    \end{align*}
  \end{proof}

\item Prove Theorem~1.2 (the additive property).

  \begin{proof}
    When \(a=b\), the proof is trivial since all are zeros.  Then, we
    prove the property holds when \(a < b\).  We can find the common
    refinement \(P = \{x_0, x_1, \dotsc, x_n\}\) of the partitions \(P_1\)
    and \(P_2\) corresponding to step functions \(s\) and \(t\)
    respectively.  Simply let \(P = P_1 \cup P_2\). Hence
    \begin{align*}
      \int_a^b s + \int_a^b t
        &= \sum_{k=1}^n s_k (x_k - x_{k-1}) + \sum_{k=1}^n t_k (x_k - x_{k-1})
      && \reason{by definition,} \\
        &= \sum_{k=1}^n \brce*{s_k (x_k - x_{k-1}) + t_k (x_k - x_{k-1})}
      && \reason{by additivity of summation,} \\
        &=\sum_{k=1}^n (s_k + t_k)(x_k - x_{k-1})
      && \reason{by distributivity,} \\
        &= \int_a^b (s + t)
      && \reason{by definition.}
    \end{align*}
    When \(a > b\), we
    have
    \begin{align*}
      \int_a^b (s + t)
        &= -\int_b^a (s + t)                            && \reason{by our convention,} \\
        &= -\paren*{\int_b^a s + \int_b^a t}            && \reason{by the fact that \(b < a\),} \\
        &= \paren*{-\int_b^a s} + \paren*{- \int_b^a t} && \reason{by distributivity,} \\
        &= \int_a^b s + \int_a^b t                      && \reason{by our convention again.} \qedhere
    \end{align*}
  \end{proof}
\end{enumerate}
\end{document}
