\documentclass{article}

\usepackage{CJKutf8, amsmath, amsthm, amsfonts, caption}
\usepackage[a4paper, top=1in, bottom=1in, left=1.25in, right=1.25in]{geometry}
\usepackage{hyperref, adjustbox, cancel}

\setlength{\parskip}{12pt}
\renewcommand{\arraystretch}{2}
\hypersetup{ colorlinks=true, linkcolor=blue }

\newcommand*{\hangpar}[2]{\hangindent=1cm \textbf{#1}\\[6pt]#2}
\newcommand*{\veq}{\ensuremath{\mathrel{\:\rotatebox{90}{=}}}}
\newcommand*{\intertexti}[1]{\intertext{\indent#1}}

\begin{document}
\begin{CJK*}{UTF8}{gkai}
\section*{第6章\ 不定积分}
\subsection*{\S 1. 原函数与不定积分} \ 

\vspace{-3em}
\textbf{定义: }若函数$f(x)$是函数$ F(x) $在开区间$ \left(a, b\right) $上的导函数,
则称函数$F(x)$为函数$f(x)$在$(a,b)$上的一个原函数.

\textbf{例: } $ f(x) = \sin x $是在$\mathbb{R}$上的一个函数, 则函数$ F(x) = -\cos x $是
$ f(x) $在$ \mathbb{R} $上的一个原函数, 因为$ f(x) = F'(x) $.

\begin{table*}[h]
	\caption*{导数表}
	\centering
	\begin{tabular}{ l l }
		原函数		&	导函数 \\
		$F(x)$		&	$f(x)$ \\
		$x^p$		&	$px^{p-1}$ \\
		$e^x$		&	$e^x$ \\
		$a^x$		&	$a^x \cdot \ln a$ \\
		$\ln |x|$	&	$\frac{1}{x}$
	\end{tabular}
	\begin{tabular}{ l l }
		原函数			&	导函数 \\
		$F(x)$			&	$f(x)$ \\
		$\log_a |x|$	&	$\frac{1}{x \cdot \ln a}$ \\
		$\sin x$		&	$\cos x$ \\
		$\cos x$		&	$-\sin x$ \\
		$\tan x$		&	$\sec^2 x$
	\end{tabular}
	\begin{tabular}{ l l }
		原函数		&	导函数 \\
		$F(x)$		&	$f(x)$ \\
		$\arcsin x$	&	$\frac{1}{\sqrt{1-x^2}}$ \\
		$\arccos x$&	$-\frac{1}{\sqrt{1-x^2}}$ \\
		$\arctan x$&	$\frac{1}{1+x^2}$ \\ \ 
	\end{tabular}
\end{table*}

\begin{table*}[h]
	\caption*{原函数表}
	\centering
	\begin{tabular}{ l l }
		函数			&	原函数 \\
		$f(x)$			&	$F(x)$ \\
		$x^p(p\ne-1)$	&	$\frac{1}{p+1} x^{p+1}$ \\
		$e^x$			&	$e^x$ \\
		$a^x$			&	$\frac{a^x}{\ln a}$ \\
		$\frac{1}{x}$	&	$\ln |x|$
	\end{tabular}
	\begin{tabular}{ l l }
		函数						&	原函数 \\
		$f(x)$						&	$F(x)$ \\
		$\sin x$					&	$-\cos x$ \\
		$\cos x$					&	$\sin x$ \\
		$\sec^2 x$					&	$\tan x$ \\
		$\frac{1}{\sqrt{1-x^2}}$	&	$\arcsin x$
	\end{tabular}
	\begin{tabular}{ l l }
		函数						&	原函数 \\
		$f(x)$						&	$F(x)$ \\
		$-\frac{1}{\sqrt{1-x^2}}$	&	$\arccos x$ \\
		$\frac{1}{1+x^2}$			&	$\arctan x$ \\ \\ \ 
	\end{tabular}
\end{table*}

\textbf{问题I: }什么样的$f(x)$在$(a,b)$上存在原函数?

\textbf{答: }\parbox[t]{5in}{
(1) $ f \in C(a,b) $, 则$ f $在$(a,b)$内一定存在原函数.(下一章, 即第7章)\\
(2) $f$在$(a,b)$不连续, 是否还有可能存在原函数?}

\textbf{反例}
\begin{gather*}
	f(x) =
	\begin{cases}
		2x \sin \frac{1}{x} - \cos \frac{1}{x}, & x \ne 0 \\
		0, & x = 0
	\end{cases} \label{E:counterExample} \tag{反例}\\
	F(x) =
	\begin{cases}
		x^2 \sin \frac{1}{x}, & x \ne 0 \\
		0, & x = 0
	\end{cases}
\end{gather*}
$F(x)$在$\mathbb{R}$上是可导函数, 所以有$F'(x) = f(x)$, \ $x \in \mathbb{R}$.

\textbf{问题II: }什么样的函数$f(x)$在$(a,b)$上没有原函数?

\textbf{答: }要回答这个问题, 我们要先回顾一下微分学中学过的Darboux定理.

\hangpar{Darboux定理: }{
若$F(x)$在$[a,b]$上可导(在$a$点右导数存在, 在b点左导数存在)且$F'(a) = \alpha$,
$F'(b) = \beta$, $\alpha \ne \beta$, 则对于任何介于$\alpha$, $\beta$的实数$\eta$,
存在$\xi \in (a,b) $使得$F'(\xi) = \eta$. (导数的介值定理) }

Darboux定理的逆否命题就说明: 不满足介值性质的函数没有原函数.

\textbf{例: }若$f(x)$在$(a,b)$上有第一类间断点, 则$f(x)$在$(a,b)$不满足介值定理,
从而在$(a,b)$上没有原函数.
\[f(x) = \begin{cases}
1, & x \ge 0 \\
-1, & x < 0
\end{cases}\]
$x = 0$是第一类间断点, $f(x)$在$(a,b)\ (a<0, b>0)$上没有原函数.

\textbf{例: }
\[f(x) = \begin{cases}
2x \sin \frac{1}{x} - \cos \frac{1}{x} + 2, & x \ne 0 \\
-2, & x = 0
\end{cases}\]
$x=0$是第二类间断点, $f(x)$在$(a,b)\ (a<0, b>0)$上不存在原函数. (cf.~\ref{E:counterExample})

\textbf{问题III: }若$f(x)$在$(a,b)$上有原函数, 有几个原函数?

\textbf{答: }有无数个.

\hangpar{不同原函数之间的关系: }{
若$F(x)$是$f(x)$在$(a,b)$内的一个原函数, 则$F(x)+C$均为$f(x)$的原函数, 并且$f(x)$的
所有原函数构成的集合为$\{F(x)+C\}$, 其中$C \in \mathbb{R}$为任意常数.}

\textbf{证明: }

\vspace{-6pt}
(i) 若$F(x)$是$f(x)$在$(a,b)$上的一个原函数, 则$F'(x)=f(x)$, \ $x\in (a,b)$. 从而$[F(x)+C]' = f(x)$,
\ $x\in (a,b)$, 所以$F(x)+C$都是$f(x)$的原函数.

(ii) 若$G(x)$是$f(x)$在$(a,b)$上的一个原函数, 则$G'(x)=f(x)=F'(x)$, $x\in (a,b)$, 从而$[G(x) - F(x)]' = 0$,
$x \in (a, b)$, 这就意味着 $G(x) - F(x) = C$, 所以$ G(x) = F(x) +C$, $x \in (a, b)$. \qed

我们称$\{F(x)+C\}$为原函数族, 只要找一个原函数为代表, 就能表示所有的原函数.

\textbf{定义: }我们把原函数族称为$f(x)$的不定积分. 记作
\[\{F(x)+C\} = \int f(x) \,dx.\]
\begin{alignat*}{3}
	F&(x) &&\xrightarrow{\text{求导}} &&f(x) \; \text{导函数} \\
	F(x) &+ C &&\xleftarrow[\text{不定积分}]{} &&f(x) \\
	&\veq \\
	\int f(&x) \, dx && &&\text{互为逆过程}
\end{alignat*}

通过上面的原函数表, 给每一个函数加上一个$C$, 就可以构成不定积分表.

\hangpar{不定积分的性质}{
(1) 若$f,g \in R[a,b]$, 则$\int [f(x)+g(x)] \,dx = \int f(x) \,dx + \int g(x) \,dx$\,; \\
(2) 若$f \in R[a,b]$, $\lambda \in \mathbb{R}$, 则$\int [\lambda f(x)] \,dx = \lambda \int f(x) \,dx$.}

\newpage
\textbf{对(1)的证明: }

\vspace{-6pt}
(i) 因为$ (F(x) + G(x))' = F'(x) + G'(x) = f(x) + g(x) $, 所以$ F(x) + G(x) $是$ f(x) + g(x) $的一个原函数.

(ii)
\begin{align*}
	\left( \int [f(x) + g(x)] \,dx \right)' = f(x) &+ g(x) \\
	& \veq \\
	\left( \int f(x) \,dx + \int g(x) \,dx \right)' = f(x) &+ g(x) \\
	\intertexti{所以}
	\int [f(x) + g(x)] \,dx = \int f(x) \,dx &+ \int g(x) \,dx.
\end{align*} \qed

\textbf{对(2)的证明: }

\vspace{-6pt}
(i) 因为$ (\lambda F(x))' = \lambda F'(x) = \lambda f(x) $, 所以$ \lambda F(x) $是$ \lambda f(x) $的一个原函数.

(ii)
\begin{align*}
	\left( \int \lambda f(x) \,dx \right)' = \lambda &f(x) \\
	& \veq \\
	\left( \lambda \int f(x) \,dx \right)' = \lambda &f(x) \\
	\intertexti{所以}
	\int \lambda f(x) \,dx = \lambda &\int f(x) \,dx.
\end{align*} \qed

性质(1)叫做加法法则, 性质(2)叫做数乘法则, 同时使用这两个法则得到:
\[ \int [\lambda f(x) + \mu g(x)] \,dx = \int \lambda f(x) \,dx + \int \mu g(x) \,dx
= \lambda \int f(x) \,dx + \mu \int g(x) \,dx. \]

所谓的减法法则, 可以看成上式的特例: 
\[ \int [f(x) - g(x)] \,dx = \int f(x) \,dx - \int g(x) \,dx \quad (\lambda = 1, \mu = -1). \]

\textbf{例1: }$ \displaystyle \int (3x^2 - 2^x) \,dx. $
\begin{align*}
	\int (3x^2 - 2^x) \,dx &= 3 \int x^2 \,dx - \int 2^x \,dx \\
	&= 3 \left(\frac{x^3}{3} + C\right) - \left(\frac{2^x}{\ln 2} + C\right) \\
	&= x^3 - \frac{2^x}{\ln 2} + {\color{red} C}.
\end{align*}

\textbf{例2: }$ \displaystyle \int \frac{x^4}{1+x^2} \, dx. $
\begin{align*}
	\intertexti{将这个代入上式}
	x^4  = (x^4 -1) + 1 &= (x^2-1)(x^2+1) + 1 \\
	\intertexti{得}
	\int \frac{x^4}{1+x^2} \, dx
	&= \int \frac{(x^2-1)(x^2+1) + 1}{1+x^2} \, dx \\
	&= \int \left(x^2 -1 + \frac{1}{1+x^2}\right) \, dx \\
	& = \int x^2 \, dx - \int dx + \int \frac{1}{1+x^2} \, dx \\
	& = \frac{x^3}{3} - x + \arctan x + C.
\end{align*}

\textbf{例3: }$ \displaystyle \int \frac{\cos 2x}{\cos x - \sin x} \, dx. $
\begin{align*}
\intertexti{将倍角公式代入}
\cos 2x = \cos^2 x - \sin^2 x &= (\cos x + \sin x)(\cos x - \sin x) \\
\int \frac{(\cos x + \sin x)\cancel{(\cos x - \sin x)}}{\cancel{\cos x - \sin x}} \, dx
&= \int (\cos x + \sin x) \, dx \\
&= \int \cos x \, dx + \int \sin x \, dx \\
&= \sin x - \cos x + C.
\end{align*}

\textbf{例4: }$ \displaystyle \int \lvert x - 1 \rvert \, dx. $
\begin{align*}
	\lvert x - 1 \rvert &= \begin{cases}
		x - 1, & x \ge 1 \\
		1 - x, & x < 1
	\end{cases} \\
	\int \lvert x - 1 \rvert \, dx &= \begin{cases}
		\int (x - 1) \, dx, & x \ge 1 \\
		\int (1 - x) \, dx, & x < 1
	\end{cases} \\
	&= \begin{cases}
		\frac{x^2}{2} - x + C_1, & x \ge 1 \\
		x - \frac{x^2}{2} + C_2, & x < 1
	\end{cases} \\
	\intertexti{为了保证$\int \lvert x - 1 \rvert \, dx$在$x=1$处可导, 则必须保证在此处连续. 所以}
	C_1 &= C_2 + 1 \\
	\intertexti{设}
	F(x) &= \begin{cases}
		\frac{x^2}{2} - x + 1, & x \ge 1 \\
		x - \frac{x^2}{2}, & x < 1
	\end{cases} \\
	\intertexti{则$ F(x) $为$ \lvert x - 1 \rvert $的一个原函数, 就有}
	\int \lvert x - 1 \rvert \, dx & = F(x) + C.
\end{align*}

\end{CJK*}
\end{document}
