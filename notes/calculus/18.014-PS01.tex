\documentclass[a4paper]{article}

\title{Solutions to Problem Set 1}
\author{L. F. \textsc{Jaw}}

\usepackage[T1]{fontenc}
\usepackage{textcomp}
\usepackage{mathtools,amssymb,amsthm}
\usepackage[hmargin=1in,vmargin=1in]{geometry}
\usepackage{graphicx,cancel}
\usepackage{hyperref}
\hypersetup{%
  colorlinks=true,
  urlcolor=[rgb]{0,0.2,0.6},
  linkcolor={.},
  bookmarksdepth=2}
\usepackage{bookmark}

\frenchspacing

\newcommand*{\parasp}{\setlength{\parskip}{10pt plus 2pt minus 3pt}}
\newcommand*{\noparasp}{\setlength{\parskip}{0pt plus 1pt}}
\newcommand*{\setparasp}[1]{\setlength{\parskip}{#1}}
\newcommand*{\pskip}{\vskip 10pt plus 2pt minus 3pt}
% \newcommand\LEFTRIGHT[3]{\left#1 #3 \right#2}
\newcommand\SetSymbol[1][]{%
  \nonscript\:#1\vert
  \allowbreak
  \nonscript\:
  \mathopen{}}
% \newcommand*{\paren}[1]{\LEFTRIGHT(){#1}}
\DeclarePairedDelimiterX{\paren}[1]{\lparen}{\rparen}{%
  \renewcommand{\mid}{\SetSymbol[\delimsize]}#1}
% \newcommand*{\brkt}[1]{\LEFTRIGHT[]{#1}}
\DeclarePairedDelimiterX{\brkt}[1]{\lbrack}{\rbrack}{%
  \renewcommand{\mid}{\SetSymbol[\delimsize]}#1}
\DeclarePairedDelimiterX{\brce}[1]{\lbrace}{\rbrace}{%
  \renewcommand{\mid}{\SetSymbol[\delimsize]}#1}
\newcommand*{\unit}[1]{\,\mathrm{#1}}
\newcommand*{\DeclareUnit}[2]{\newcommand*{#1}{\unit{#2}}}
\DeclareUnit{\cm}{cm}
% \renewcommand*{\m}{\unit{m}}
\DeclareUnit{\m}{m}
\DeclareUnit{\kg}{kg}
\DeclareUnit{\s}{s}
\newcommand*{\R}{\mathbb{R}}
\newcommand*{\Z}{\mathbb{Z}}
% \newcommand*{\Rp}{(0,+\infty)}
% \newcommand*{\Rm}{(-\infty,0)}
\newcommand*{\deduce}{\mathrel{\Downarrow}}
% \newcommand*{\abs}[1]{\left\lvert #1 \right\rvert}
\DeclarePairedDelimiter{\abs}{\lvert}{\rvert}
% \newcommand*{\ceil}[1]{\left\lceil#1\right\rceil}
\DeclarePairedDelimiter{\ceil}{\lceil}{\rceil}
\newcommand*{\textop}[1]{\mathop{\text{#1}}}

\newcommand*{\enumparen}[1]{(\makebox[0.6em][c]{#1})}
\renewcommand{\labelenumii}{\enumparen{\theenumii}}

\DeclareMathOperator{\arccosh}{arccosh}
\DeclareMathOperator{\gammaf}{\Gamma}
\DeclareMathOperator{\var}{var}
\DeclareMathOperator{\Ber}{Bernoulli}
\DeclareMathOperator{\Cov}{Cov}
\DeclareMathOperator{\E}{E}
\def\argmax{\qopname\relax m{arg\,max}}
\DeclarePairedDelimiterXPP{\Eb}[1]{\E}{\lbrack}{\rbrack}{}{%
  \renewcommand{\mid}{\SetSymbol[\delimsize]}#1}
\DeclarePairedDelimiterXPP{\varp}[1]{\var}{\lparen}{\rparen}{}{%
  \renewcommand{\mid}{\SetSymbol[\delimsize]}#1}
\DeclarePairedDelimiterXPP{\Covp}[1]{\Cov}{\lparen}{\rparen}{}{%
  \renewcommand{\mid}{\SetSymbol[\delimsize]}#1}
\DeclarePairedDelimiterXPP{\expp}[1]{\exp}{\lbrace}{\rbrace}{}{#1}
\renewcommand*{\Pr}{\mathop{P}}
\DeclarePairedDelimiterXPP{\Prp}[1]{\Pr}{\lparen}{\rparen}{}{%
  \renewcommand{\mid}{\SetSymbol[\delimsize]}#1}
\newcommand*{\pnorm}{\mathop{\Phi}}
\DeclarePairedDelimiterXPP{\pnormp}[1]{\pnorm}{\lparen}{\rparen}{}{#1}
\newcommand*{\dnorm}{\mathop{\varphi}}
\DeclarePairedDelimiterXPP{\dnormp}[1]{\dnorm}{\lparen}{\rparen}{}{#1}
\newcommand*{\qnorm}{\mathop{\Phi^{-1}}}
%\newcommand*{\diff}{\mathop{}\!d}
\newcommand*{\diff}{\mathop{}\!\mathit{d}}
%\newcommand*{\diff}{\mathop{}\!\mathrm{d}}
\newcommand*{\dx}{\diff x}
\newcommand*{\dy}{\diff y}
\newcommand*{\dz}{\diff z}
\newcommand*{\ds}{\diff s}
\newcommand*{\dt}{\diff t}
\newcommand*{\du}{\diff u}
\newcommand*{\dv}{\diff v}
\newcommand*{\dtheta}{\diff \theta}
\newcommand*{\dd}[2][]{\frac{\diff#1}{\diff#2}}
\newcommand*{\ddx}{\frac{\diff}{\dx}}
\newcommand*{\ddt}{\frac{\diff}{\dt}}
\newcommand*{\ddy}{\dd y}
\newcommand*{\ddtheta}{\frac{\diff}{\dtheta}}
\newcommand*{\ddz}{\dd z}
\newcommand*{\fwdf}{\mathop{}\!\Delta}
\newcommand*{\dydx}{\frac\dy\dx}
\newcommand*{\pdpd}[2][]{\frac{\partial#1}{\partial#2}}
\newcommand*{\pdpdx}{\frac\partial{\partial x}}
\newcommand*{\pdpdy}{\frac\partial{\partial y}}
\newcommand*{\pdpdz}{\frac\partial{\partial z}}
\newcommand*{\pdpdu}{\frac\partial{\partial u}}
\newcommand*{\pdpdv}{\frac\partial{\partial v}}
\newcommand*{\pdpdt}{\frac\partial{\partial t}}
\newcommand*{\pdzpdx}{\frac{\partial z}{\partial x}}
\newcommand*{\pdzpdy}{\frac{\partial z}{\partial y}}
\newcommand*{\pdzpdt}{\frac{\partial z}{\partial t}}
\newcommand*{\pdxpdt}{\frac{\partial x}{\partial t}}
\newcommand*{\pdypdt}{\frac{\partial y}{\partial t}}

% \usepackage[lite,subscriptcorrection,nofontinfo]{mtpro2}
\usepackage{fontspec}

\setmainfont{Palatino Linotype}[Ligatures=TeX,Numbers=OldStyle]
\setmonofont{Source Code Pro}
% \usepackage[integrals]{wasysym}
\usepackage{fontawesome}

\usepackage[math-style=TeX]{unicode-math}
\setmathfont{TeX Gyre Pagella Math}

\usepackage{microtype}

\let\reason\text
\let\vect\symbf

\AtBeginDocument{%
  % \renewcommand{\perp}{\mathrel{\bot}}
  \let\leq\leqslant
  \let\le\leq
  \let\geq\geqslant
  \let\ge\geq}


\begin{document}
\maketitle

This is my own solution to the problem set 1 of
\href{https://ocw.mit.edu/courses/mathematics/18-014-calculus-with-theory-fall-2010/assignments/}{18.014}.
The first three problems are from Apostol's \emph{Calculus} and the next
three problems are from Munkres's Course Note A (1: 18, 20, 43; 9--10).

\begin{enumerate}
\item Prove Theorem I.11: If \(ab=0\), then \(a=0\) or \(b=0\).

  \begin{proof}
    If not, then \(a \ne 0\) and \(b \ne 0\), which implies
    \begin{align*}
      1 &= 1 \cdot 1           && \reason{by the identity axiom,} \\
        &= (aa^{-1}) (bb^{-1}) && \reason{by the inverse axiom,} \\
        &= (a^{-1}a) (bb^{-1}) && \reason{by commutativity,} \\
        &= a (a(bb^{-1}))      && \reason{by associativity,} \\
        &= a ((ab)b^{-1})      && \reason{by associativity,} \\
        &= a (0 \cdot b^{-1})  && \reason{by assumption,} \\
        &= a \cdot 0           && \reason{by Theorem I.6,} \\
        &= 0                   && \reason{by Theorem I.6.}
    \end{align*}
    But we know by Axiom 4 that \(1 \ne 0\).  Thus, either \(a = 0\) or
    \(b = 0\).
  \end{proof}

\item Prove Theorem I.25: If \(a < c\) and \(b < d\), then \(a+b < c+d\).

  \begin{proof}
    We have
    \begin{align*}
      a + b &< c + b && \reason{by Theorem I.18,} \\
            &= b + c && \reason{by commutativity,} \\
            &< d + c && \reason{by Theorem I.18,} \\
            &= c + d && \reason{by commutativity.} \qedhere
    \end{align*}
  \end{proof}

\item Prove that \(\abs*{\abs x - \abs y} \le \abs{x - y}\).

  There are two ways to prove this.  One is logical and the other is
  algebraic.

  \begin{proof}
    If \(\abs x - \abs y \ge 0\), then
    \[
      \abs*{\abs x - \abs y} = \abs x - \abs y \le \abs{x - y}.
    \]
    If \(\abs x - \abs y < 0\), then
    \[
      \abs*{\abs x - \abs y} = - (\abs x - \abs y) = \abs y - \abs x \le \abs{y - x} = \abs{x - y}.
    \]
    In all cases, the inequality holds.
  \end{proof}

  \begin{proof}
    We have
    \begin{align*}
      xy &\le \abs{xy} = \abs x \abs y         && \iff \\
      -2\abs x\abs y
         &\le -2xy                             && \iff \\
      \abs x^2 - 2\abs x\abs y + \abs y^2
         &\le x^2 - 2xy + y^2                  && \iff \\
      \paren{\abs x - \abs y}^2
         &\le \paren{x - y}^2                  && \iff \\
      \sqrt{\paren{\abs x - \abs y}^2}
         &\le \sqrt{\paren{x - y}^2}           && \iff \\
      \abs*{\abs x - \abs y} &\le \abs{x - y}. && \qedhere
    \end{align*}
  \end{proof}

\item Prove Theorem 6: If \(a\) and \(b\) are in
  \(\mathbf{P}\), so is \(ab\).

  \begin{proof}
    We choose a fixed positive integer \(a\) and then prove this theorem
    by induction on \(b\).  First, \(a \cdot 1\) is in \(\mathbf{P}\),
    since \(a \cdot 1 = a\) by the identity axiom and \(a\) is in
    \(\mathbf{P}\).  Now suppose \(b\) is a positive integer such that
    \(ab\) is in \(\mathbf{P}\).  Then \(a(b+1) = ab + a \).  By the
    induction hypothesis, \(ab\) is a positive integer.  And \(a\) is also
    a positive integer.  We have already shown that \(P\) is closed under
    addition.  Thus, \(ab + a\) is also a positive integer, which means
    \(a(b+1)\) is in \(\mathbf{P}\).  This proves that \(\mathbf{P}\) is
    also closed under multiplication.
  \end{proof}
  
\item Prove Theorem 12: \(a^n b^n = \paren{ab}^n\), where \(a\) and \(b\) are any real numbers and \(n\)
  is a positive integer.
\end{enumerate}

\end{document}
