\documentclass[a4paper]{article}

\title{Solutions to Recitation~7}
\author{L. F. \textsc{Jaw}}

\usepackage[T1]{fontenc}
\usepackage{textcomp}
\usepackage{mathtools,amssymb,amsthm}
\usepackage[hmargin=1in,vmargin=1in]{geometry}
\usepackage{graphicx,cancel}
\usepackage{hyperref}
\hypersetup{%
  colorlinks=true,
  urlcolor=[rgb]{0,0.2,0.6},
  linkcolor={.},
  bookmarksdepth=2}
\usepackage{bookmark}

\frenchspacing

\newcommand*{\parasp}{\setlength{\parskip}{10pt plus 2pt minus 3pt}}
\newcommand*{\noparasp}{\setlength{\parskip}{0pt plus 1pt}}
\newcommand*{\setparasp}[1]{\setlength{\parskip}{#1}}
\newcommand*{\pskip}{\vskip 10pt plus 2pt minus 3pt}
% \newcommand\LEFTRIGHT[3]{\left#1 #3 \right#2}
\newcommand\SetSymbol[1][]{%
  \nonscript\:#1\vert
  \allowbreak
  \nonscript\:
  \mathopen{}}
% \newcommand*{\paren}[1]{\LEFTRIGHT(){#1}}
\DeclarePairedDelimiterX{\paren}[1]{\lparen}{\rparen}{%
  \renewcommand{\mid}{\SetSymbol[\delimsize]}#1}
% \newcommand*{\brkt}[1]{\LEFTRIGHT[]{#1}}
\DeclarePairedDelimiterX{\brkt}[1]{\lbrack}{\rbrack}{%
  \renewcommand{\mid}{\SetSymbol[\delimsize]}#1}
\DeclarePairedDelimiterX{\brce}[1]{\lbrace}{\rbrace}{%
  \renewcommand{\mid}{\SetSymbol[\delimsize]}#1}
\newcommand*{\unit}[1]{\,\mathrm{#1}}
\newcommand*{\DeclareUnit}[2]{\newcommand*{#1}{\unit{#2}}}
\DeclareUnit{\cm}{cm}
% \renewcommand*{\m}{\unit{m}}
\DeclareUnit{\m}{m}
\DeclareUnit{\kg}{kg}
\DeclareUnit{\s}{s}
\newcommand*{\R}{\mathbb{R}}
\newcommand*{\Z}{\mathbb{Z}}
% \newcommand*{\Rp}{(0,+\infty)}
% \newcommand*{\Rm}{(-\infty,0)}
\newcommand*{\deduce}{\mathrel{\Downarrow}}
% \newcommand*{\abs}[1]{\left\lvert #1 \right\rvert}
\DeclarePairedDelimiter{\abs}{\lvert}{\rvert}
% \newcommand*{\ceil}[1]{\left\lceil#1\right\rceil}
\DeclarePairedDelimiter{\ceil}{\lceil}{\rceil}
\newcommand*{\textop}[1]{\mathop{\text{#1}}}

\newcommand*{\enumparen}[1]{(\makebox[0.6em][c]{#1})}
\renewcommand{\labelenumii}{\enumparen{\theenumii}}

\DeclareMathOperator{\arccosh}{arccosh}
\DeclareMathOperator{\gammaf}{\Gamma}
\DeclareMathOperator{\var}{var}
\DeclareMathOperator{\Ber}{Bernoulli}
\DeclareMathOperator{\Cov}{Cov}
\DeclareMathOperator{\E}{E}
\def\argmax{\qopname\relax m{arg\,max}}
\DeclarePairedDelimiterXPP{\Eb}[1]{\E}{\lbrack}{\rbrack}{}{%
  \renewcommand{\mid}{\SetSymbol[\delimsize]}#1}
\DeclarePairedDelimiterXPP{\varp}[1]{\var}{\lparen}{\rparen}{}{%
  \renewcommand{\mid}{\SetSymbol[\delimsize]}#1}
\DeclarePairedDelimiterXPP{\Covp}[1]{\Cov}{\lparen}{\rparen}{}{%
  \renewcommand{\mid}{\SetSymbol[\delimsize]}#1}
\DeclarePairedDelimiterXPP{\expp}[1]{\exp}{\lbrace}{\rbrace}{}{#1}
\renewcommand*{\Pr}{\mathop{P}}
\DeclarePairedDelimiterXPP{\Prp}[1]{\Pr}{\lparen}{\rparen}{}{%
  \renewcommand{\mid}{\SetSymbol[\delimsize]}#1}
\newcommand*{\pnorm}{\mathop{\Phi}}
\DeclarePairedDelimiterXPP{\pnormp}[1]{\pnorm}{\lparen}{\rparen}{}{#1}
\newcommand*{\dnorm}{\mathop{\varphi}}
\DeclarePairedDelimiterXPP{\dnormp}[1]{\dnorm}{\lparen}{\rparen}{}{#1}
\newcommand*{\qnorm}{\mathop{\Phi^{-1}}}
%\newcommand*{\diff}{\mathop{}\!d}
\newcommand*{\diff}{\mathop{}\!\mathit{d}}
%\newcommand*{\diff}{\mathop{}\!\mathrm{d}}
\newcommand*{\dx}{\diff x}
\newcommand*{\dy}{\diff y}
\newcommand*{\dz}{\diff z}
\newcommand*{\ds}{\diff s}
\newcommand*{\dt}{\diff t}
\newcommand*{\du}{\diff u}
\newcommand*{\dv}{\diff v}
\newcommand*{\dtheta}{\diff \theta}
\newcommand*{\dd}[2][]{\frac{\diff#1}{\diff#2}}
\newcommand*{\ddx}{\frac{\diff}{\dx}}
\newcommand*{\ddt}{\frac{\diff}{\dt}}
\newcommand*{\ddy}{\dd y}
\newcommand*{\ddtheta}{\frac{\diff}{\dtheta}}
\newcommand*{\ddz}{\dd z}
\newcommand*{\fwdf}{\mathop{}\!\Delta}
\newcommand*{\dydx}{\frac\dy\dx}
\newcommand*{\pdpd}[2][]{\frac{\partial#1}{\partial#2}}
\newcommand*{\pdpdx}{\frac\partial{\partial x}}
\newcommand*{\pdpdy}{\frac\partial{\partial y}}
\newcommand*{\pdpdz}{\frac\partial{\partial z}}
\newcommand*{\pdpdu}{\frac\partial{\partial u}}
\newcommand*{\pdpdv}{\frac\partial{\partial v}}
\newcommand*{\pdpdt}{\frac\partial{\partial t}}
\newcommand*{\pdzpdx}{\frac{\partial z}{\partial x}}
\newcommand*{\pdzpdy}{\frac{\partial z}{\partial y}}
\newcommand*{\pdzpdt}{\frac{\partial z}{\partial t}}
\newcommand*{\pdxpdt}{\frac{\partial x}{\partial t}}
\newcommand*{\pdypdt}{\frac{\partial y}{\partial t}}

% \usepackage[lite,subscriptcorrection,nofontinfo]{mtpro2}
\usepackage{fontspec}

\setmainfont{Palatino Linotype}[Ligatures=TeX,Numbers=OldStyle]
\setmonofont{Source Code Pro}
% \usepackage[integrals]{wasysym}
\usepackage{fontawesome}

\usepackage[math-style=TeX]{unicode-math}
\setmathfont{TeX Gyre Pagella Math}

\usepackage{microtype}


\usepackage{tikz}
\usetikzlibrary{positioning,shapes,automata,external}
\tikzexternalize[mode=list and make,prefix=figures/]
% \tikzexternaldisable

\let\reason\text
\let\vect\symbf

\AtBeginDocument{%
  % \renewcommand{\perp}{\mathrel{\bot}}
  \let\leq\leqslant
  \let\le\leq
  \let\geq\geqslant
  \let\ge\geq}


\begin{document}
\maketitle

This is my own solution to the Recitation~7 from
\href{https://ocw.mit.edu/courses/mathematics/18-014-calculus-with-theory-fall-2010/recitations/}{18.014},
which includes Exercise~4, 10, and 15 from Apostol's
\textit{Calculus} (1: 70--71).

\begin{enumerate}
\item
  \begin{enumerate}
  \item If \(n\) is a positive integer, prove that \(\int_0^n [t] \dt = n(n-1)/2\).

    \begin{proof}
      Let \(P = \brce{0, 1, \dotsc, n}\).  Then \(P\) is partition for
      \([t]\) on \([0, n]\) and \(s_k = k-1\) on each open subinterval.  Hence
      \begin{align*}
        \int_0^n [t] \dt
          &= \sum_{k=1}^n s_k (x_k - x_{k-1})  && \reason{by definition,} \\
          &= \sum_{k=1}^n (k-1)\brce{k - (k-1)} && \reason{by our choice of partion,} \\
          &= 0 + \sum_{k=1}^{n-1} k && \reason{by simple algebra and reindexing,} \\
          &= \frac{n(n-1)}{2}. && \qedhere
      \end{align*}
    \end{proof}

  \item If \(f(x) = \int_0^x [t] \dt\) for \(x \ge 0\), draw the graph of
    \(f\) over the interval \([0, 4]\).

    \begin{figure}[H]
      \centering
      \tikzsetnextfilename{18.014-rec07.1b}
      \begin{tikzpicture}
        \draw[->] (-1.5,0) -- (5,0) node[anchor=west] {\(x\)};
        \draw[->] (0,-1.5) -- (0,6.5) node[anchor=south] {\(y\)};
        \draw[thick] plot[mark=*,mark options={scale=0.5}]
                     coordinates{(0,0) (1,0) (2,1) (3,3) (4,6)};
        \draw[-] (-1,-0.1) node[anchor=north]                    {\(-1\)} -- (-1,0.1);
        \draw[-] (0,-0.1)  node[anchor=north east,xshift=-1.5pt] {\(0\)}  -- (0,0.1);
        \draw[-] (1,-0.1)  node[anchor=north]                    {\(1\)}  -- (1,0.1);
        \draw[-] (2,-0.1)  node[anchor=north]                    {\(2\)}  -- (2,0.1);
        \draw[-] (3,-0.1)  node[anchor=north]                    {\(3\)}  -- (3,0.1);
        \draw[-] (4,-0.1)  node[anchor=north]                    {\(4\)}  -- (4,0.1);
        \draw[-] (-0.1,-1) node[anchor=east,xshift=1.5pt]        {\(-1\)} -- (0.1,-1);
        \draw[-] (-0.1,1)  node[anchor=east,xshift=1.5pt]        {\(1\)}  -- (0.1,1);
        \draw[-] (-0.1,2)  node[anchor=east,xshift=1.5pt]        {\(2\)}  -- (0.1,2);
        \draw[-] (-0.1,3)  node[anchor=east,xshift=1.5pt]        {\(3\)}  -- (0.1,3);
        \draw[-] (-0.1,4)  node[anchor=east,xshift=1.5pt]        {\(4\)}  -- (0.1,4);
        \draw[-] (-0.1,5)  node[anchor=east,xshift=1.5pt]        {\(5\)}  -- (0.1,5);
        \draw[-] (-0.1,6)  node[anchor=east,xshift=1.5pt]        {\(6\)}  -- (0.1,6);
      \end{tikzpicture}
    \end{figure}
  \end{enumerate}

\item Given a positive integer \(p\).  A step function \(s\) is defined on
  the interval \([0, p]\) as follows: \(\fn s(x) = (-1)^n n\) if \(x\) lies in
  the interval \(n \le x < n+1\), where \(n = 0, 1, 2, \dotsc, p-1\);
  \(s(p) = 0\).  Let \(f(p) = \int_0^p s(x) \dx\).
  \begin{enumerate}
  \item Calculate \(f(3)\), \(f(4)\), and \(f(\,f(3))\).

    It's easy to see that
    \[
      f(p) = \int_0^p \fn s(x) \dx = \sum_{k=1}^n (-1)^{k-1}(k-1).
    \]
    Then we have
    \begin{gather*}
      f(3) = 0 - 1 + 2 = 1, \\
      f(4) = f(3) - 3 = -2, \\
      \intertext{and}
      f(\,f(3)) = f(1) = 0.
    \end{gather*}

  \item For what value (or values) of \(p\) is \(|\,f(p)| = 7\)?

    Since \(p\) is assumed to be a positive integer, we have
    \[
      f(p) = \sum_{k=1}^n (-1)^{k-1}(k-1) = (-1)^{p+1} [\tfrac p2].
    \]
    Thus \(|\,f(p)| = 7\) if and only if \(f(p) = \pm7\).  It is only
    when \(p = 14\) or \(p = 15\) that the original equation is satisfied.

    As an exercise, we can derive a formula without the greatest-integer
    function by summation by parts.  The result is simply
    \[
      \sum_{k=1}^n (-1)^{k-1}(k-1) = \tfrac12 \brce{(p-\tfrac12)(1-(-1)^p) - p}.
    \]
  \end{enumerate}

\item Prove Theorem~1.5 (the comparison theorem).

  \begin{proof}
    Notice that it is assumed that \(a < b\).  Since \(s\) and \(t\) are
    step functions, so are \(-s\) and \(t-s\).  Furthermore,
    \(\fn t(x) - \fn s(x) > 0\) for every \(x\) in \([a, b]\).  Let
    \(u = t - s\).  Then there exists a partition
    \(\brce{x_0, x_1, \dotsc, x_n}\) such that \(u\) is constant on each
    open subinterval.  Let \(u_k\) denote the constant on the \(k\)th
    subinterval.  Hence
    \begin{align*}
      \int_a^b (t - s)
        &= \sum_{k=1}^n u_k (x_k - x_{k-1}) && \reason{by definition,} \\
        &> 0 && \reason{by Axiom~7 and Theorem~I.25.}
    \end{align*}
    By linearity of integral, we have
    \[
      \int_a^b t - \int_a^b s = \int_a^b (t - s) > 0. \qedhere
    \]
  \end{proof}
\end{enumerate}
\end{document}
