\documentclass[a4paper]{article}

\title{Solutions to Practice Exam~1}
\author{L. F. \textsc{Jaw}}

\usepackage[T1]{fontenc}
\usepackage{textcomp}
\usepackage{mathtools,amssymb,amsthm}
\usepackage[hmargin=1in,vmargin=1in]{geometry}
\usepackage{graphicx,cancel}
\usepackage{hyperref}
\hypersetup{%
  colorlinks=true,
  urlcolor=[rgb]{0,0.2,0.6},
  linkcolor={.},
  bookmarksdepth=2}
\usepackage{bookmark}

\frenchspacing

\newcommand*{\parasp}{\setlength{\parskip}{10pt plus 2pt minus 3pt}}
\newcommand*{\noparasp}{\setlength{\parskip}{0pt plus 1pt}}
\newcommand*{\setparasp}[1]{\setlength{\parskip}{#1}}
\newcommand*{\pskip}{\vskip 10pt plus 2pt minus 3pt}
% \newcommand\LEFTRIGHT[3]{\left#1 #3 \right#2}
\newcommand\SetSymbol[1][]{%
  \nonscript\:#1\vert
  \allowbreak
  \nonscript\:
  \mathopen{}}
% \newcommand*{\paren}[1]{\LEFTRIGHT(){#1}}
\DeclarePairedDelimiterX{\paren}[1]{\lparen}{\rparen}{%
  \renewcommand{\mid}{\SetSymbol[\delimsize]}#1}
% \newcommand*{\brkt}[1]{\LEFTRIGHT[]{#1}}
\DeclarePairedDelimiterX{\brkt}[1]{\lbrack}{\rbrack}{%
  \renewcommand{\mid}{\SetSymbol[\delimsize]}#1}
\DeclarePairedDelimiterX{\brce}[1]{\lbrace}{\rbrace}{%
  \renewcommand{\mid}{\SetSymbol[\delimsize]}#1}
\newcommand*{\unit}[1]{\,\mathrm{#1}}
\newcommand*{\DeclareUnit}[2]{\newcommand*{#1}{\unit{#2}}}
\DeclareUnit{\cm}{cm}
% \renewcommand*{\m}{\unit{m}}
\DeclareUnit{\m}{m}
\DeclareUnit{\kg}{kg}
\DeclareUnit{\s}{s}
\newcommand*{\R}{\mathbb{R}}
\newcommand*{\Z}{\mathbb{Z}}
% \newcommand*{\Rp}{(0,+\infty)}
% \newcommand*{\Rm}{(-\infty,0)}
\newcommand*{\deduce}{\mathrel{\Downarrow}}
% \newcommand*{\abs}[1]{\left\lvert #1 \right\rvert}
\DeclarePairedDelimiter{\abs}{\lvert}{\rvert}
% \newcommand*{\ceil}[1]{\left\lceil#1\right\rceil}
\DeclarePairedDelimiter{\ceil}{\lceil}{\rceil}
\newcommand*{\textop}[1]{\mathop{\text{#1}}}

\newcommand*{\enumparen}[1]{(\makebox[0.6em][c]{#1})}
\renewcommand{\labelenumii}{\enumparen{\theenumii}}

\DeclareMathOperator{\arccosh}{arccosh}
\DeclareMathOperator{\gammaf}{\Gamma}
\DeclareMathOperator{\var}{var}
\DeclareMathOperator{\Ber}{Bernoulli}
\DeclareMathOperator{\Cov}{Cov}
\DeclareMathOperator{\E}{E}
\def\argmax{\qopname\relax m{arg\,max}}
\DeclarePairedDelimiterXPP{\Eb}[1]{\E}{\lbrack}{\rbrack}{}{%
  \renewcommand{\mid}{\SetSymbol[\delimsize]}#1}
\DeclarePairedDelimiterXPP{\varp}[1]{\var}{\lparen}{\rparen}{}{%
  \renewcommand{\mid}{\SetSymbol[\delimsize]}#1}
\DeclarePairedDelimiterXPP{\Covp}[1]{\Cov}{\lparen}{\rparen}{}{%
  \renewcommand{\mid}{\SetSymbol[\delimsize]}#1}
\DeclarePairedDelimiterXPP{\expp}[1]{\exp}{\lbrace}{\rbrace}{}{#1}
\renewcommand*{\Pr}{\mathop{P}}
\DeclarePairedDelimiterXPP{\Prp}[1]{\Pr}{\lparen}{\rparen}{}{%
  \renewcommand{\mid}{\SetSymbol[\delimsize]}#1}
\newcommand*{\pnorm}{\mathop{\Phi}}
\DeclarePairedDelimiterXPP{\pnormp}[1]{\pnorm}{\lparen}{\rparen}{}{#1}
\newcommand*{\dnorm}{\mathop{\varphi}}
\DeclarePairedDelimiterXPP{\dnormp}[1]{\dnorm}{\lparen}{\rparen}{}{#1}
\newcommand*{\qnorm}{\mathop{\Phi^{-1}}}
%\newcommand*{\diff}{\mathop{}\!d}
\newcommand*{\diff}{\mathop{}\!\mathit{d}}
%\newcommand*{\diff}{\mathop{}\!\mathrm{d}}
\newcommand*{\dx}{\diff x}
\newcommand*{\dy}{\diff y}
\newcommand*{\dz}{\diff z}
\newcommand*{\ds}{\diff s}
\newcommand*{\dt}{\diff t}
\newcommand*{\du}{\diff u}
\newcommand*{\dv}{\diff v}
\newcommand*{\dtheta}{\diff \theta}
\newcommand*{\dd}[2][]{\frac{\diff#1}{\diff#2}}
\newcommand*{\ddx}{\frac{\diff}{\dx}}
\newcommand*{\ddt}{\frac{\diff}{\dt}}
\newcommand*{\ddy}{\dd y}
\newcommand*{\ddtheta}{\frac{\diff}{\dtheta}}
\newcommand*{\ddz}{\dd z}
\newcommand*{\fwdf}{\mathop{}\!\Delta}
\newcommand*{\dydx}{\frac\dy\dx}
\newcommand*{\pdpd}[2][]{\frac{\partial#1}{\partial#2}}
\newcommand*{\pdpdx}{\frac\partial{\partial x}}
\newcommand*{\pdpdy}{\frac\partial{\partial y}}
\newcommand*{\pdpdz}{\frac\partial{\partial z}}
\newcommand*{\pdpdu}{\frac\partial{\partial u}}
\newcommand*{\pdpdv}{\frac\partial{\partial v}}
\newcommand*{\pdpdt}{\frac\partial{\partial t}}
\newcommand*{\pdzpdx}{\frac{\partial z}{\partial x}}
\newcommand*{\pdzpdy}{\frac{\partial z}{\partial y}}
\newcommand*{\pdzpdt}{\frac{\partial z}{\partial t}}
\newcommand*{\pdxpdt}{\frac{\partial x}{\partial t}}
\newcommand*{\pdypdt}{\frac{\partial y}{\partial t}}

% \usepackage[lite,subscriptcorrection,nofontinfo]{mtpro2}
\usepackage{fontspec}

\setmainfont{Palatino Linotype}[Ligatures=TeX,Numbers=OldStyle]
\setmonofont{Source Code Pro}
% \usepackage[integrals]{wasysym}
\usepackage{fontawesome}

\usepackage[math-style=TeX]{unicode-math}
\setmathfont{TeX Gyre Pagella Math}

\usepackage{microtype}

\let\reason\text
\let\vect\symbf

\AtBeginDocument{%
  % \renewcommand{\perp}{\mathrel{\bot}}
  \let\leq\leqslant
  \let\le\leq
  \let\geq\geqslant
  \let\ge\geq}


\begin{document}
\maketitle

This is my own solution to Practice Exam~1 of
\href{https://ocw.mit.edu/courses/mathematics/18-014-calculus-with-theory-fall-2010/exams/}{18.014}.

\begin{enumerate}
\item Compute \(\int_{99}^{103} \paren{2x-198}^2 [\sqrt{x-99}] \dx\)
  where here \([x]\) is defined to be the largest integer \(\le x\).

  We have
  \begin{align*}
    \int_{99}^{103} \paren{2x-198}^2 [\sqrt{x-99}] \dx
      &= \int_0^4 \paren{2x}^2 [\sqrt x] \dx \\
      &= \int_0^1 0 \dx + 4 \int_1^4 x^2 \dx \\
      &= 4 \cdot \frac{4^3 - 1^3}{3} \\
      &= 84.
  \end{align*}

\item Let \(S\) be a square pyramid with base area \(r^2\) and height
  \(h\).  Using Cavalieri's Theorem, determine the volume of the
  pyramid.

  By elementary geometry, we have
  \[
    a_S(x) = \frac{(h-x)^2 r^2}{h^2}.
  \]
  Then
  \begin{align*}
    v(S) &= \int_0^h a_S(x) \dx \\
         &= \int_0^h \frac{(h-x)^2 r^2}{h^2} \dx \\
         &= \frac{r^2}{h^2} \int_0^h (h-x)^2 \dx \\
         &= \frac{r^2}{h^2} \int_{-h}^0 x^2 \dx \\
         &= \frac{r^2}{h^2} \cdot \frac{h^3}{3}
           = \frac{r^2 h}{3}.
  \end{align*}

\item Let \(f\) be an integrable function on \([0, 1]\).  Prove that
  \(\abs{\,f}\) is integrable on \([0, 1]\).

  \begin{proof}
    Since \(f\) is integrable on \([0, 1]\), for any \(\epsilon > 0\), we
    can always find step functions \(s\) and \(t\) such that
    \(s \le f \le t\) and
    \[
      \int_0^1 t - \int_0^1 s < \epsilon.
    \]
    Let \(P = \brce{x_0, x_1, \dotsc, x_n}\) be a common partition for \(s\)
    and \(t\).  Denote \(s(x) = s_k\) and \(t(x) = t_k\) on every open
    interval \((x_{k-1}, x_k)\).  We construct two other step functions
    \begin{gather*}
      s'(x) =
      \begin{cases}
        s(x), & s(x) \ge 0, \\
        -t(x), & t(x) \le 0,
      \end{cases}
      \qand
      t'(x) =
      \begin{cases}
        t(x), & s(x) \ge 0, \\
        -s(x), & t(x) \le 0.
      \end{cases}
    \end{gather*}
    It is easy to verify that \(s' \le \abs{\,f} \le t'\) and
    \[
      \int_0^1 t' - \int_0^1 s' = \int_0^1 t - \int_0^1 s < \epsilon.
    \]
    This means that \(\abs{\,f}\) also satisfies the Riemann condition and
    is thus integrable.
  \end{proof}

\item The well-ordering principle states that every non-empty subset of
  the natural numbers has a least element.  Prove the well-ordering
  principle implies the principle of mathematical induction.

  \begin{proof}
    Let \(T\) be the set of all positive integers not in \(S\).  Suppose
    \(T\) is nonempty.  By the well-ordering principle, \(T\) has a smallest
    member.  Let \(m\) denote such a member.  The fact that \(m \notin S \)
    implies \(m \ne 1\).  It cannot be the case that \(m < 1\) since \(1\)
    is the smallest positive integer.  Thus, \(m > 1\).  Then \(m-1\) is
    another integer since \(\mathbf{Z}\) is closed under subtraction.  And
    it is positive since \(m > 1\) implies \(m-1>0\).  It cannot be in
    \(S\).  If so, then \(m\) will be in \(S\), which contradicts our
    definition of \(T\).  This means \(m-1\) is another positive integer not
    in \(S\) and is thus in \(T\).  Notice that \(m-1\) is smaller than
    \(m\).  This contradicts the well-ordering principle.  Thus, \(T\) must
    be empty.  This is to say that \(S\) contains all positive integers.
  \end{proof}

\item Suppose \(\lim_{x \to p^+} f(x) = \lim_{x \to p^-} f(x) = A\).
  Prove \(\lim_{x \to p} f(x) = A\).

  \begin{proof}
    For any \(\epsilon > 0\), there exist \(\delta_1, \delta_2 > 0\) such
    that
    \begin{gather*}
      \abs{\,f(x) - A} < \epsilon \txt{whenener} 0 < x - p < \delta_1 \\
      \iand
      \abs{\,f(x) - A} < \epsilon \txt{whenener} 0 < p - x < \delta_2.
    \end{gather*}
    Let \(\delta = \min\brce{\delta_1, \delta_2}\).  Then
    \[
      \abs{\,f(x) - A} < \epsilon \txt{whenener} 0 < \abs{x - p} < \delta.  \qedhere
    \]
  \end{proof}
\end{enumerate}
\end{document}
