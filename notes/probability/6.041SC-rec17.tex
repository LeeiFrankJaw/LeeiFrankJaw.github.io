\documentclass[a4paper]{article}

\usepackage[T1]{fontenc}
\usepackage{textcomp}
\usepackage{mathtools,amssymb,amsthm}
\usepackage[hmargin=1in,vmargin=1in]{geometry}
\usepackage{graphicx,cancel}
\usepackage[math-style=TeX]{unicode-math}
% \usepackage[lite,subscriptcorrection,nofontinfo]{mtpro2}
\usepackage{fontspec}
\usepackage[colorlinks=true,allcolors=blue]{hyperref}

\defaultfontfeatures{Ligatures=TeX,Numbers=OldStyle}
\setmainfont{Palatino Linotype}
\setmathfont{TeX Gyre Pagella Math}
%\usepackage[integrals]{wasysym}
\frenchspacing

\newcommand*{\parasp}{\setlength{\parskip}{10pt plus 2pt minus 3pt}}
\newcommand*{\noparasp}{\setlength{\parskip}{0pt plus 1pt}}
\newcommand*{\setparasp}[1]{\setlength{\parskip}{#1}}
\newcommand*{\pskip}{\vskip 10pt plus 2pt minus 3pt}
\newcommand\LEFTRIGHT[3]{\left#1 #3 \right#2}
\newcommand*{\paren}[1]{\LEFTRIGHT(){#1}}
\newcommand*{\brkt}[1]{\LEFTRIGHT[]{#1}}
\newcommand*{\unit}[1]{\,\mathrm{#1}}
\newcommand*{\DeclareUnit}[2]{\newcommand*{#1}{\unit{#2}}}
\DeclareUnit{\cm}{cm}
% \renewcommand*{\m}{\unit{m}}
\DeclareUnit{\m}{m}
\DeclareUnit{\kg}{kg}
\DeclareUnit{\s}{s}
\newcommand*{\R}{\mathbb{R}}
\newcommand*{\Rp}{(0,+\infty)}
\newcommand*{\Rm}{(-\infty,0)}
\newcommand*{\deduce}{\mathrel{\Downarrow}}
\newcommand*{\abs}[1]{\left\lvert #1 \right\rvert}
\newcommand*{\reason}[1]{\langle \, \text{#1} \, \rangle}

\newcounter{ListCounter}
\newenvironment{enumerate*}%
{\begin{list}%
        {\arabic{ListCounter}.}%
        {\usecounter{ListCounter}%
            \setlength{\topsep}{1.5pt}%
            \setlength{\itemsep}{1.5pt} } }%
    {\end{list}}

\DeclareMathOperator{\arccosh}{arccosh}
\DeclareMathOperator{\gammaf}{\Gamma}
\DeclareMathOperator{\var}{var}
\DeclareMathOperator{\Ber}{Bernoulli}
\DeclareMathOperator{\Cov}{Cov}
\DeclareMathOperator{\E}{E}
%\newcommand*{\diff}{\mathop{}\!d}
\newcommand*{\diff}{\mathop{}\!\mathit{d}}
%\newcommand*{\diff}{\mathop{}\!\mathrm{d}}
\newcommand*{\dx}{\diff x}
\newcommand*{\dy}{\diff y}
\newcommand*{\dz}{\diff z}
\newcommand*{\dt}{\diff t}
\newcommand*{\du}{\diff u}
\newcommand*{\dv}{\diff v}
\newcommand*{\dtheta}{\diff \theta}
\newcommand*{\ddx}{\frac{\diff}{\dx}}
\newcommand*{\fwdf}{\mathop{}\!\Delta}
\newcommand*{\dydx}{\frac\dy\dx}
\newcommand*{\pdpdx}{\frac\partial{\partial x}}
\newcommand*{\pdpdy}{\frac\partial{\partial y}}
\newcommand*{\pdpdz}{\frac\partial{\partial z}}
\newcommand*{\pdpdu}{\frac\partial{\partial u}}
\newcommand*{\pdpdv}{\frac\partial{\partial v}}
\newcommand*{\pdpdt}{\frac\partial{\partial t}}
\newcommand*{\pdzpdx}{\frac{\partial z}{\partial x}}
\newcommand*{\pdzpdy}{\frac{\partial z}{\partial y}}
\newcommand*{\pdzpdt}{\frac{\partial z}{\partial t}}
\newcommand*{\pdxpdt}{\frac{\partial x}{\partial t}}
\newcommand*{\pdypdt}{\frac{\partial y}{\partial t}}

\AtBeginDocument{%
	\renewcommand{\perp}{\mathrel{\bot}}
	\let\leq\leqslant
	\let\le\leq
        \let\geq\geqslant
        \let\ge\geq}

\newcommand*{\Prb}[1]{\section*{Problem #1}}
\newcommand{\Q}[1]{\textbf{Question:} #1}
\newcommand*{\A}[1]{\textbf{Answer:} #1}
\newcommand{\Prblm}[3]{\Prb{#1} \Q{#2} \\[6pt] \A{#3}}


\title{Solutions to Recitation 17}
\author{Lei Zhao}

\hyphenpenalty=687

\begin{document}
\maketitle

This is my own solutions to the problems from recitation 17 of
\href{https://ocw.mit.edu/courses/electrical-engineering-and-computer-science/6-041sc-probabilistic-systems-analysis-and-applied-probability-fall-2013/unit-iii/lecture-15/}{6.041\textsc{sc}}.

\begin{enumerate}
\item Iwana Passe is taking a multiple-choice exam.  You may assume that the
  number of questions is infinite.  \emph{Simultaneously, but independently}, her
  conscious and subconscious faculties are generating answers for her, each
  in a Poisson manner.  (Her conscious and subconscious are always working
  on different questions.)  Conscious responses are generated at the rate
  \(λ_c\) responses per minute.  Subconscious responses are generated at the
  rate \(λ_s\) responses per minute.  Assume \(λ_c \ne λ_s\).  Each
  conscious response is an independent Bernoulli trial with probability
  \(p_c\) of being correct.  Similarly, each subconscious response is an
  independent Bernoulli trial with probability \(p_s\) of being correct.
  Iwana responds only once to each question, and you can assume that her
  time for recording these conscious and subconscious responses is
  negligible.
  \begin{enumerate} \parasp
  \item Determine \(p_K(k)\), the probability mass function for the number
    of conscious responses Iwana makes in an interval of \(T\) minutes.

    \[p_K(k) = \frac{(\lambda_c T)^k}{k!} e^{-\lambda_c T}.\]

  \item If we pick any question to which Iwana has responded, what is the
    probability that her answer to that question:
    \begin{enumerate} \parasp
    \item represents a conscious response?

      \[\frac{\lambda_c}{\lambda_c + \lambda_s}.\]

    \item represents a conscious correct response?
      \[\frac{\lambda_c p_c}{\lambda_c + \lambda_s}.\]
    \end{enumerate}

  \item If we pick an interval of \(T\) minutes, what is the probability
    that in that interval Iwana will make exactly \(r\) conscious responses
    and \(s\) subconscious responses?

    \[\frac{\lambda_c^r \lambda_s^s T^{r+s}}{r!s!} e^{-(\lambda_c + \lambda_s) T}.\]

  \item Determine the probability density function for random variable
    \(X\), where \(X\) is the time from the start of the exam until Iwana
    makes her first conscious response which is preceded by at least one
    subconscious response.

    \[\frac{\lambda_c \lambda_s}{\lambda_c - \lambda_s} (e^{-\lambda_sx} - e^{-\lambda_cx}).\]

  \end{enumerate}

\item Shem, a local policeman, drives from intersection to intersection in
  times that are independent and all exponentially distributed with
  parameter \(λ\).  At each intersection he observes (and reports) a car
  accident with probability \(p\).  (This activity does not slow his driving
  at all.)  Independently of all else, Shem receives extremely brief radio
  calls in a Poisson manner with an average rate of \(μ\) calls per hour.
  \begin{enumerate} \parasp
  \item Determine the \textsc{pmf} for \(N\), the number of intersections
    Shem visits up to and including the one where he reports his first
    accident.

    \[p_N(n) = (1-p)^{n-1}p.\]

  \item Determine the \textsc{pdf} for \(Q\), the length of time Shem drives between
    reporting accidents.

    \[f_Q(q) = \lambda p e^{-\lambda p q}.\]

  \item What is the \textsc{pmf} for \(M\), the number of accidents which Shem reports in
    two hours?

    \[p_M(m) = \frac{(2\lambda p)^m}{m!} e^{-2\lambda}.\]

  \item What is the \textsc{pmf} for \(K\), the number of accidents Shem reports between
    his receipt of two successive radio calls?

    \[p_K(k) = \paren{\frac{\lambda p}{\mu + \lambda p}}^k \frac{\mu}{\mu + \lambda p}.\]

  \item We observe Shem at a random instant long after his shift has begun.
    Let \(W\) be the total time from Shem’s last radio call until his next
    radio call.  What is the \textsc{pdf} of \(W\)?

    \[f_W(w) = \mu^2 w e^{-\mu w} .\]
  \end{enumerate}

\item \emph{Random incidence in an Erlang arrival process}

  Consider an arrival process in which the interarrival times are
  independent Erlang random variables of order \(2\), with mean
  \(2/λ\).  Assume that the arrival process has been ongoing for a
  very long time.  An external observer arrives at a given time \(t\).
  Find the \textsc{pdf} of the length of the interarrival interval
  that contains \(t\).

  Let \(X\) denote the length of the interarrival interval that
  contains \(t\).  The random variable \(X\) can be decomposed into three
  component \(X_1\), \(X_2\), and \(X_3\), each of which has an exponential distribution with
  parameter \(\lambda\).  Thus, by convolution, we have

  \begin{align*}
    f_{X_1+X_2}(x)
    &= \int_0^x f_{X_1}(t) \, f_{X_2}(x - t) \dt \\
    &= \int_0^x \lambda e^{-\lambda t} \, \lambda e^{-\lambda(x - t)} \dt \\
    &= \lambda^2 e^{-\lambda x} \int_0^x \dt \\
    &= \lambda^2 x e^{-\lambda x} \\
    \intertext{and}
    f_{X}(x)
    &= \int_0^x f_{X_1+X_2}(t) \, f_{X_3}(x - t) \dt \\
    &= \int_0^x \lambda^2 t e^{-\lambda t} \, \lambda e^{-\lambda(x - t)} \dt \\
    &= \lambda^3 e^{-\lambda x} \int_0^x t \dt \\
    &= \frac{\lambda^3 x^2}{2} e^{-\lambda x}.
  \end{align*}
  Hence, the \textsc{pdf} of \(X\) is an Erlang distribution of order \(3\) with parameter \(\lambda\).
\end{enumerate}

\end{document}