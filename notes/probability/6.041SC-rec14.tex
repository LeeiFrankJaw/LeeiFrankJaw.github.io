\documentclass[a4paper]{article}

\usepackage[T1]{fontenc}
\usepackage{textcomp}
\usepackage{mathtools,amssymb,amsthm}
\usepackage[hmargin=1in,vmargin=1in]{geometry}
\usepackage{graphicx,cancel}
\usepackage[math-style=TeX]{unicode-math}
% \usepackage[lite,subscriptcorrection,nofontinfo]{mtpro2}
\usepackage{fontspec}
\usepackage[colorlinks=true,allcolors=blue]{hyperref}

\defaultfontfeatures{Ligatures=TeX,Numbers=OldStyle}
\setmainfont{Palatino Linotype}
\setmathfont{TeX Gyre Pagella Math}
%\usepackage[integrals]{wasysym}
\frenchspacing

\newcommand*{\parasp}{\setlength{\parskip}{10pt plus 2pt minus 3pt}}
\newcommand*{\noparasp}{\setlength{\parskip}{0pt plus 1pt}}
\newcommand*{\setparasp}[1]{\setlength{\parskip}{#1}}
\newcommand*{\pskip}{\vskip 10pt plus 2pt minus 3pt}
\newcommand\LEFTRIGHT[3]{\left#1 #3 \right#2}
\newcommand*{\paren}[1]{\LEFTRIGHT(){#1}}
\newcommand*{\brkt}[1]{\LEFTRIGHT[]{#1}}
\newcommand*{\unit}[1]{\,\mathrm{#1}}
\newcommand*{\DeclareUnit}[2]{\newcommand*{#1}{\unit{#2}}}
\DeclareUnit{\cm}{cm}
% \renewcommand*{\m}{\unit{m}}
\DeclareUnit{\m}{m}
\DeclareUnit{\kg}{kg}
\DeclareUnit{\s}{s}
\newcommand*{\R}{\mathbb{R}}
\newcommand*{\Rp}{(0,+\infty)}
\newcommand*{\Rm}{(-\infty,0)}
\newcommand*{\deduce}{\mathrel{\Downarrow}}
\newcommand*{\abs}[1]{\left\lvert #1 \right\rvert}
\newcommand*{\reason}[1]{\langle \, \text{#1} \, \rangle}

\newcounter{ListCounter}
\newenvironment{enumerate*}%
{\begin{list}%
        {\arabic{ListCounter}.}%
        {\usecounter{ListCounter}%
            \setlength{\topsep}{1.5pt}%
            \setlength{\itemsep}{1.5pt} } }%
    {\end{list}}

\DeclareMathOperator{\arccosh}{arccosh}
\DeclareMathOperator{\gammaf}{\Gamma}
\DeclareMathOperator{\var}{var}
\DeclareMathOperator{\Ber}{Bernoulli}
\DeclareMathOperator{\Cov}{Cov}
\DeclareMathOperator{\E}{E}
%\newcommand*{\diff}{\mathop{}\!d}
\newcommand*{\diff}{\mathop{}\!\mathit{d}}
%\newcommand*{\diff}{\mathop{}\!\mathrm{d}}
\newcommand*{\dx}{\diff x}
\newcommand*{\dy}{\diff y}
\newcommand*{\dz}{\diff z}
\newcommand*{\dt}{\diff t}
\newcommand*{\du}{\diff u}
\newcommand*{\dv}{\diff v}
\newcommand*{\dtheta}{\diff \theta}
\newcommand*{\ddx}{\frac{\diff}{\dx}}
\newcommand*{\fwdf}{\mathop{}\!\Delta}
\newcommand*{\dydx}{\frac\dy\dx}
\newcommand*{\pdpdx}{\frac\partial{\partial x}}
\newcommand*{\pdpdy}{\frac\partial{\partial y}}
\newcommand*{\pdpdz}{\frac\partial{\partial z}}
\newcommand*{\pdpdu}{\frac\partial{\partial u}}
\newcommand*{\pdpdv}{\frac\partial{\partial v}}
\newcommand*{\pdpdt}{\frac\partial{\partial t}}
\newcommand*{\pdzpdx}{\frac{\partial z}{\partial x}}
\newcommand*{\pdzpdy}{\frac{\partial z}{\partial y}}
\newcommand*{\pdzpdt}{\frac{\partial z}{\partial t}}
\newcommand*{\pdxpdt}{\frac{\partial x}{\partial t}}
\newcommand*{\pdypdt}{\frac{\partial y}{\partial t}}

\AtBeginDocument{%
	\renewcommand{\perp}{\mathrel{\bot}}
	\let\leq\leqslant
	\let\le\leq
        \let\geq\geqslant
        \let\ge\geq}

\newcommand*{\Prb}[1]{\section*{Problem #1}}
\newcommand{\Q}[1]{\textbf{Question:} #1}
\newcommand*{\A}[1]{\textbf{Answer:} #1}
\newcommand{\Prblm}[3]{\Prb{#1} \Q{#2} \\[6pt] \A{#3}}


\title{Solutions to Recitation 14}
\author{Lei Zhao}

\begin{document}
\maketitle

This is my own solutions to the recitation 14 problems from
\href{https://ocw.mit.edu/courses/electrical-engineering-and-computer-science/6-041sc-probabilistic-systems-analysis-and-applied-probability-fall-2013/unit-iii/lecture-13/}{6.041\textsc{sc}}.

\begin{enumerate}
\item You are visiting the rainforest, but unfortunately your insect
  repellent has run out.  As a result, at each second, a mosquito
  lands on your neck with probability \(0.5\).  If one lands, with
  probability \(0.2\) it bites you, and with probability \(0.8\) it
  never bothers you, independently of other mosquitoes.

  \begin{enumerate} \parasp
  \item What is the expected time between successive mosquito bites?
    What is the variance of the time between successive mosquito
    bites?

    At each second, the probability that a mosquito lands on your neck
    and bites you is \(0.5 \times 0.2 = 0.1\).  As the length of first
    string of losing days discussed in the lecture, the time between
    successive mosquito bites have the same model.  It is a geometric
    variable with parameter \(0.1\). So the expected time is
    \(1/0.1 = 10\) and the variance is \((1-0.1)/0.1^2 = 90\).

  \item In addition, a tick lands on your neck with probability
    \(0.1\).  If one lands, with probability \(0.7\) it bites you, and
    with probability \(0.3\), it never bothers you, independently of
    other ticks and mosquitoes.  Now, what is expected time between
    successive bug bites?  What is the variance of the time between
    successive bug bites?

    This is a merging of two Bernoulli processes into another Bernoulli
    process.  At each second, the probability that a bug bites you is
    \(1 - (1 - 0.5 \times 0.2)(1 - 0.1 \times 0.7) = 0.163\).  The time
    between successive bug bites is a geometric random variable with
    parameter \(0.163\).  Thus the expecation is \(1/0.163 \approx 6.135\)
    and the variance \((1-0.163)/0.163^2 \approx 31.503\).
  \end{enumerate}

\item Al performs an experiment comprising a series of independent
  trials.  On each trial, he simultaneously flips a set of three fair
  coins.

  \begin{enumerate} \parasp
  \item Given that Al has just had a trial with \(3\) \emph{tails},
    what is the probability that both of the next two trials will also
    have this result?

    Since these trials are independent, what happened in the past has no
    influence on what will happen in the future.  So the probability is
    simply \((1/2)^3 (1/2)^3 = 1/64\).

  \item Whenever all three coins land on the same side in any given
    trial, Al calls the trial a success. \noparasp

    \begin{enumerate} \parasp
    \item Find the \textsc{pmf} for \(K\), the number of trials up to, but
      \emph{not} including, the second success.

      At each trial, the probability of success is \(2 \times (1/2)^3 = 1/4\).
      Thus the \textsc{pmf} is

      \[ p_K(k) = \left(\frac34\right)^k \frac14 = \frac{3^k}{4^{k+1}}. \]

    \item Find the expectation and variance of \(M\), the number of tails that
      occur \emph{before} the first success.

      Let \(G\) be the geometric random variable with parameter \(1/4\) and
      \(X_i\) is the number of tails in trial \(i\).  Trials are independent
      from each other and so are \(X_i\).  Conditioned on \(G\), we have
      \begin{align*}
        \E[M \mid G = g] &= \E[\sum_{i=1}^{g-1} X_i \mid G = g] \\
                         &= \sum_{i=1}^{g-1} \E[X_i \mid G = g] \\
                         &= \sum_{i=1}^{g-1} \sum_{j=1}^2 j \, p_{X_i \mid G}(j \mid g) \\
                         &= \sum_{i=1}^{g-1} \left(\frac{{3 \choose 1} (1/2)^3}{1 - 2 (1/2)^3} + 2\frac{{3 \choose 2} (1/2)^3}{1 - 2 (1/2)^3}\right) \\
                         &= \sum_{i=1}^{g-1} \frac32 = \frac32 (g-1),
      \end{align*}
      and
      \begin{align*}
        \var(M \mid G = g) &= \var(\sum_{i=1}^{g-1} X_i \mid G = g) \\
                           &= \sum_{i=1}^{g-1} \var(X_i \mid G = g) \\
                           &= \sum_{i=1}^{g-1} \left(\left(1- \frac32\right)^2 \frac12 + \left(2 - \frac32\right)^2 \frac12\right) \\
        &= \sum_{i=1}^{g-1} \frac14 = \frac14 (g-1).
      \end{align*}
      By applying law of iterated expectation, we have
      \begin{align*}
        \E[M] &= \E[\E[M \mid G]] \\
              &= \E[\frac32 (G-1)] \\
              &= \frac32 (\E[G] - 1) = \frac92.
      \end{align*}
      By applying law of total variance, we have
      \begin{align*}
        \var(M) &= \E[\var(M \mid G)] + \var(\E[M \mid G]) \\
                &= \E[\frac14 (G-1)] + \var(\frac32 (G-1)) \\
                &= \frac14 (\E[G] - 1) + \paren{\frac32}^2 \var(G) \\
                &= \frac34 + 27 = \frac{111}{4} = 27.75.
      \end{align*}
    \end{enumerate}
  \item Bob conducts an experiment like Al’s, except that he uses 4 coins
    for the first trial, and then he obeys the following rule: Whenever
    all of the coins land on the same side in a trial, Bob permanently
    removes one coin from the experiment and continues with the trials.
    He follows this rule until the \emph{third} time he removes a coin, at
    which point the experiment ceases. Find \(\E[N]\), where \(N\) is the
    number of trials in Bob’s experiment. \parasp

    It can be easily shown that \(N\) is a sum of three geometric random
    variables \(G_1\), \(G_2\), \(G_3\) with parameter \(1/8\), \(1/4\),
    and \(1/2\) respectively. Thus
    \(\E[N] = \E[G_1+ G_2 + G_3] = \E[G_1] + \E[G_2] + \E[G_3] = 8 + 4 + 2
    = 14\).
  \end{enumerate}
\item Suppose there are \(n\) papers in a drawer.  You draw a paper and
  sign it, and then, instead of filing it away, you place the paper back
  into the drawer.  If any paper is equally likely to be drawn each time,
  independent of all other draws, what is the expected number of papers
  that you will draw before signing all \(n\) papers?  You may leave your
  answer in the form of a summation. \parasp

  Let \(N\) be such the number of papers.  This also can be seen as a sum
  of \(n\) geomeric random variables \(N_1\), \(N_2\), \dots, \(N_n\) with
  parameter \(n/n\), \((n-1)/n\), \dots, \(1/n\).  Notice that the first
  random variable is degenerate.  The expected number is
  \begin{align*}
    \E[N] &= \E[N_1 + N_2 + \dots + N_n] \\
          &= \E[N_1]+ \E[N_2] + \dots + \E[N_n] \\
          &= \frac{n}{n} + \frac{n}{n-1} + \dots + \frac{n}{1} \\
          &= n \sum_{k=1}^n \frac1k = n \, H_n.
  \end{align*}
\end{enumerate}
\end{document}