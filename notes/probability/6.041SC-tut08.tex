\documentclass[a4paper]{article}

\usepackage[T1]{fontenc}
\usepackage{textcomp}
\usepackage{mathtools,amssymb,amsthm}
\usepackage[hmargin=1in,vmargin=1in]{geometry}
\usepackage{graphicx,cancel}
\usepackage[math-style=TeX]{unicode-math}
% \usepackage[lite,subscriptcorrection,nofontinfo]{mtpro2}
\usepackage{fontspec}
\usepackage[colorlinks=true,allcolors=blue]{hyperref}

\defaultfontfeatures{Ligatures=TeX,Numbers=OldStyle}
\setmainfont{Palatino Linotype}
\setmathfont{TeX Gyre Pagella Math}
%\usepackage[integrals]{wasysym}
\frenchspacing

\newcommand*{\parasp}{\setlength{\parskip}{10pt plus 2pt minus 3pt}}
\newcommand*{\noparasp}{\setlength{\parskip}{0pt plus 1pt}}
\newcommand*{\setparasp}[1]{\setlength{\parskip}{#1}}
\newcommand*{\pskip}{\vskip 10pt plus 2pt minus 3pt}
\newcommand\LEFTRIGHT[3]{\left#1 #3 \right#2}
\newcommand*{\paren}[1]{\LEFTRIGHT(){#1}}
\newcommand*{\brkt}[1]{\LEFTRIGHT[]{#1}}
\newcommand*{\unit}[1]{\,\mathrm{#1}}
\newcommand*{\DeclareUnit}[2]{\newcommand*{#1}{\unit{#2}}}
\DeclareUnit{\cm}{cm}
% \renewcommand*{\m}{\unit{m}}
\DeclareUnit{\m}{m}
\DeclareUnit{\kg}{kg}
\DeclareUnit{\s}{s}
\newcommand*{\R}{\mathbb{R}}
\newcommand*{\Rp}{(0,+\infty)}
\newcommand*{\Rm}{(-\infty,0)}
\newcommand*{\deduce}{\mathrel{\Downarrow}}
\newcommand*{\abs}[1]{\left\lvert #1 \right\rvert}
\newcommand*{\reason}[1]{\langle \, \text{#1} \, \rangle}

\newcounter{ListCounter}
\newenvironment{enumerate*}%
{\begin{list}%
        {\arabic{ListCounter}.}%
        {\usecounter{ListCounter}%
            \setlength{\topsep}{1.5pt}%
            \setlength{\itemsep}{1.5pt} } }%
    {\end{list}}

\DeclareMathOperator{\arccosh}{arccosh}
\DeclareMathOperator{\gammaf}{\Gamma}
\DeclareMathOperator{\var}{var}
\DeclareMathOperator{\Ber}{Bernoulli}
\DeclareMathOperator{\Cov}{Cov}
\DeclareMathOperator{\E}{E}
%\newcommand*{\diff}{\mathop{}\!d}
\newcommand*{\diff}{\mathop{}\!\mathit{d}}
%\newcommand*{\diff}{\mathop{}\!\mathrm{d}}
\newcommand*{\dx}{\diff x}
\newcommand*{\dy}{\diff y}
\newcommand*{\dz}{\diff z}
\newcommand*{\dt}{\diff t}
\newcommand*{\du}{\diff u}
\newcommand*{\dv}{\diff v}
\newcommand*{\dtheta}{\diff \theta}
\newcommand*{\ddx}{\frac{\diff}{\dx}}
\newcommand*{\fwdf}{\mathop{}\!\Delta}
\newcommand*{\dydx}{\frac\dy\dx}
\newcommand*{\pdpdx}{\frac\partial{\partial x}}
\newcommand*{\pdpdy}{\frac\partial{\partial y}}
\newcommand*{\pdpdz}{\frac\partial{\partial z}}
\newcommand*{\pdpdu}{\frac\partial{\partial u}}
\newcommand*{\pdpdv}{\frac\partial{\partial v}}
\newcommand*{\pdpdt}{\frac\partial{\partial t}}
\newcommand*{\pdzpdx}{\frac{\partial z}{\partial x}}
\newcommand*{\pdzpdy}{\frac{\partial z}{\partial y}}
\newcommand*{\pdzpdt}{\frac{\partial z}{\partial t}}
\newcommand*{\pdxpdt}{\frac{\partial x}{\partial t}}
\newcommand*{\pdypdt}{\frac{\partial y}{\partial t}}

\AtBeginDocument{%
	\renewcommand{\perp}{\mathrel{\bot}}
	\let\leq\leqslant
	\let\le\leq
        \let\geq\geqslant
        \let\ge\geq}

\newcommand*{\Prb}[1]{\section*{Problem #1}}
\newcommand{\Q}[1]{\textbf{Question:} #1}
\newcommand*{\A}[1]{\textbf{Answer:} #1}
\newcommand{\Prblm}[3]{\Prb{#1} \Q{#2} \\[6pt] \A{#3}}


\title{Solutions to Tutorial 8}
\author{Lei Zhao}

\begin{document}
\maketitle

This is my own solutions to the problems from tutorial 8 of
\href{https://ocw.mit.edu/courses/electrical-engineering-and-computer-science/6-041sc-probabilistic-systems-analysis-and-applied-probability-fall-2013/unit-iii/lecture-15/}{6.041\textsc{sc}}.

\begin{enumerate}
\item Type A, B, and C items are placed in a common
  buffer, each type arriving as part of an inde­pendent Poisson
  process with average arrival rates, respectively, of \(a\), \(b\),
  and \(c\) items per minute.  For the first four parts of this
  problem, assume the buffer is discharged immediately whenever it
  contains a total of ten items.
  \begin{enumerate}
  \item What is the probability that, of the first ten items to arrive
    at the buffer, only the first and one other are type A?

    \[\frac{a}{a+b+c} \binom91 \frac{a}{a+b+c} \paren{\frac{b+c}{a+b+c}}^8 = \frac{9 a^2 (b+c)^8}{(a+b+c)^{10}}.\]

  \item What is the probability that any particular discharge of the
    buffer contains five times as many type A items as type B items?

    \[\binom{10}{5, 1, 4}\paren{\frac{a}{a+b+c}}^5 \paren{\frac{b}{a+b+c}} \paren{\frac{c}{a+b+c}}^4 + \paren{\frac{c}{a+b+c}}^{10} = \frac{1260 \, a^5 b c^4 + c^{10}}{(a+b+c)^{10}}.\]

  \item Determine the \textsc{pdf}, expectation, and variance for the total
    time between consecutive discharges of the buffer.

    Let \(T\) be the time between consecutive discharges of the buffer.  Then,
    \begin{gather*}
      f_T(t) = \frac{(a+b+c)^{10} t^9}{9!} e^{-(a+b+c)t}, \\
      E[T] = \frac{10}{a+b+c}, \\
      \intertext{and}
      \var(T) = \frac{10}{(a+b+c)^2}.
    \end{gather*}

  \item Determine the probability that exactly two of each of the
    three item types arrive at the buffer input during any particular
    five minute interval.

    \[\paren{\frac{(5a)^2}{2} e^{-5a}}\paren{\frac{(5c)^2}{2} e^{-5c}}\paren{\frac{(5c)^2}{2} e^{-5c}} = \frac{(125abc)^2}{8} e^{-5(a+b+c)}.\]
  \end{enumerate}

\item A store opens at \(t = 0\) and \emph{potential} customers arrive
  in a Poisson manner at an average arrival rate of \(λ\) potential
  customers per hour.  As long as the store is open, and independently
  of all other events, each particular potential customer becomes an
  \emph{actual} customer with probability \(p\).  The store closes as
  soon as ten actual customers have arrived.
  \begin{enumerate}
  \item What is the probability that exactly three of the first five
    potential customers become actual customers?

    \[\binom53 \, p^3 (1-p)^2.\]

  \item What is the probability that the fifth potential customer to
    arrive becomes the third actual customer?

    \[\binom42 \, p^3 (1-p)^2.\]

  \item What is the \textsc{pdf} and expected value for \(L\), the
    duration of the interval from store opening to store closing?

    \begin{gather*}
      f_L(t) = \frac{(\lambda p)^{10} t^9}{9!} e^{-\lambda p t}, \\
      \intertext{and}
      E[L] = \frac{10}{\lambda p}.
    \end{gather*}

  \item Given only that exactly three of the first five potential
    customers became actual customers, what is the conditional
    expected value of the \emph{total} time the store is open?

    \[\frac{5}{λ} + \frac{7}{λp}.\]
    
  \item Considering only customers arriving between \(t = 0\) and the
    closing of the store, what is the probability that no two actual
    customers arrive within \(τ\) time units of each other?

    \[e^{-9λpτ}.\]
  \end{enumerate}
  
\item Consider a Poisson process with parameter \(λ\), and an
  independent random variable \(T\), which is exponential with
  parameter \(ν\).  Find the \textsc{pmf} of the number of Poisson
  arrivals during the time interval \([0, T]\).

  Let \(K\) be the number of Poisson arrivals.  Then,
  \begin{align*}
    p_K(k) &= \int_0^{+\infty} f_T(t) \, p_{K \mid T}(k \mid t) \dt \\
           &= \int_0^{+\infty} νe^{-νt} \, \frac{(λt)^k}{k!} e^{-λt} \dt \\
           &= \frac{νλ^k}{k!} \int_0^{+\infty} t^k e^{-(λ+ν)t} \dt \\
           &= \frac{νλ^k}{k!} \left.\paren{\sum_{k=0}^k \frac{k!t^i}{i!(λ+ν)^{k+1-i}} e^{-(λ+ν)t}}\right\vert_{+\infty}^0 \\
           &= \frac{νλ^k}{(λ+ν)^{k+1}}.
  \end{align*}
\end{enumerate}

\end{document}