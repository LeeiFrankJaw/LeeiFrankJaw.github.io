\documentclass[a4paper]{article}

\usepackage[T1]{fontenc}
\usepackage{textcomp}
\usepackage{mathtools,amssymb,amsthm}
\usepackage[hmargin=1in,vmargin=1in]{geometry}
\usepackage{graphicx,cancel}
\usepackage[math-style=TeX]{unicode-math}
% \usepackage[lite,subscriptcorrection,nofontinfo]{mtpro2}
\usepackage{fontspec}
\usepackage[colorlinks=true,allcolors=blue]{hyperref}

\defaultfontfeatures{Ligatures=TeX,Numbers=OldStyle}
\setmainfont{Palatino Linotype}
\setmathfont{TeX Gyre Pagella Math}
%\usepackage[integrals]{wasysym}
\frenchspacing

\newcommand*{\parasp}{\setlength{\parskip}{10pt plus 2pt minus 3pt}}
\newcommand*{\noparasp}{\setlength{\parskip}{0pt plus 1pt}}
\newcommand*{\setparasp}[1]{\setlength{\parskip}{#1}}
\newcommand*{\pskip}{\vskip 10pt plus 2pt minus 3pt}
\newcommand\LEFTRIGHT[3]{\left#1 #3 \right#2}
\newcommand*{\paren}[1]{\LEFTRIGHT(){#1}}
\newcommand*{\brkt}[1]{\LEFTRIGHT[]{#1}}
\newcommand*{\unit}[1]{\,\mathrm{#1}}
\newcommand*{\DeclareUnit}[2]{\newcommand*{#1}{\unit{#2}}}
\DeclareUnit{\cm}{cm}
% \renewcommand*{\m}{\unit{m}}
\DeclareUnit{\m}{m}
\DeclareUnit{\kg}{kg}
\DeclareUnit{\s}{s}
\newcommand*{\R}{\mathbb{R}}
\newcommand*{\Rp}{(0,+\infty)}
\newcommand*{\Rm}{(-\infty,0)}
\newcommand*{\deduce}{\mathrel{\Downarrow}}
\newcommand*{\abs}[1]{\left\lvert #1 \right\rvert}
\newcommand*{\reason}[1]{\langle \, \text{#1} \, \rangle}

\newcounter{ListCounter}
\newenvironment{enumerate*}%
{\begin{list}%
        {\arabic{ListCounter}.}%
        {\usecounter{ListCounter}%
            \setlength{\topsep}{1.5pt}%
            \setlength{\itemsep}{1.5pt} } }%
    {\end{list}}

\DeclareMathOperator{\arccosh}{arccosh}
\DeclareMathOperator{\gammaf}{\Gamma}
\DeclareMathOperator{\var}{var}
\DeclareMathOperator{\Ber}{Bernoulli}
\DeclareMathOperator{\Cov}{Cov}
\DeclareMathOperator{\E}{E}
%\newcommand*{\diff}{\mathop{}\!d}
\newcommand*{\diff}{\mathop{}\!\mathit{d}}
%\newcommand*{\diff}{\mathop{}\!\mathrm{d}}
\newcommand*{\dx}{\diff x}
\newcommand*{\dy}{\diff y}
\newcommand*{\dz}{\diff z}
\newcommand*{\dt}{\diff t}
\newcommand*{\du}{\diff u}
\newcommand*{\dv}{\diff v}
\newcommand*{\dtheta}{\diff \theta}
\newcommand*{\ddx}{\frac{\diff}{\dx}}
\newcommand*{\fwdf}{\mathop{}\!\Delta}
\newcommand*{\dydx}{\frac\dy\dx}
\newcommand*{\pdpdx}{\frac\partial{\partial x}}
\newcommand*{\pdpdy}{\frac\partial{\partial y}}
\newcommand*{\pdpdz}{\frac\partial{\partial z}}
\newcommand*{\pdpdu}{\frac\partial{\partial u}}
\newcommand*{\pdpdv}{\frac\partial{\partial v}}
\newcommand*{\pdpdt}{\frac\partial{\partial t}}
\newcommand*{\pdzpdx}{\frac{\partial z}{\partial x}}
\newcommand*{\pdzpdy}{\frac{\partial z}{\partial y}}
\newcommand*{\pdzpdt}{\frac{\partial z}{\partial t}}
\newcommand*{\pdxpdt}{\frac{\partial x}{\partial t}}
\newcommand*{\pdypdt}{\frac{\partial y}{\partial t}}

\AtBeginDocument{%
	\renewcommand{\perp}{\mathrel{\bot}}
	\let\leq\leqslant
	\let\le\leq
        \let\geq\geqslant
        \let\ge\geq}

\newcommand*{\Prb}[1]{\section*{Problem #1}}
\newcommand{\Q}[1]{\textbf{Question:} #1}
\newcommand*{\A}[1]{\textbf{Answer:} #1}
\newcommand{\Prblm}[3]{\Prb{#1} \Q{#2} \\[6pt] \A{#3}}


\title{Solutions to Tutorial 7}
\author{Lei Zhao}

\begin{document}
\maketitle

This is my own solutions to the problems from tutorial 7 of
\href{https://ocw.mit.edu/courses/electrical-engineering-and-computer-science/6-041sc-probabilistic-systems-analysis-and-applied-probability-fall-2013/unit-iii/lecture-13/}{6.041\textsc{sc}}.

\begin{enumerate}
\item Alice and Bob alternate playing at the casino table.  (Alice
  starts and plays at odd times \(i = 1, 3, \dots\); Bob plays at even
  times \(i = 2, 4, \dots\) .)  At each time \(i\), the net gain of
  whoever is playing is a random variable \(G_i\) with the following
  \textsc{pmf}:
  \[p_G(g) =
    \begin{cases}
      1/3, & g = -2, \\
      1/2, & g = 1,  \\
      1/6, & g = 3,  \\
      0,       & \text{otherwise}.
    \end{cases}\]
  Assume that the net gains at different times are independent.  We
  refer to an outcome of \(−2\) as a “loss.”
  \begin{enumerate} \parasp
  \item They keep gambling until the first time where a loss by Bob
    immediately follows a loss by Alice.  Write down the \textsc{pmf} of the
    total number of rounds played.  (A round consists of two plays,
    one by Alice and then one by Bob.)

    The probability that any round terminates the gambling is
    \(1/3 \cdot 1/3 = 1/9\).  The total number of rounds played is a
    geometric random variable (denoted by \(K\)) with parameter \(1/9\), whose the
    \textsc{pmf} is
    \[p_K(k) = \paren{\frac89}^{k-1} \frac19.\]

  \item Write down the \textsc{pmf} for \(Z\), defined as the time at
    which Bob has his third loss.

    The random variable \(Z\) has the Pascal distribution of order
    \(3\) with parameter \(1/3\), whose \textsc{pmf} is
    \[p_Z(z) = {z-1 \choose 2} \paren{\frac13}^3\paren{\frac23}^{z-3}.\]

  \item Let \(N\) be the number of rounds until each one of them has
    won at least once.  Find \(\E[N]\).

    We can view the process as a two-step process.  The first step is
    that the gambling continues until any one of them has won.  Given
    the first step, the second step is that the gambling continues
    until the other player has won.  We use \(W_1\) and \(W_2\) to denote the
    number of rounds during the first and second steps respectively.

    At each round, the probablity that Alice wins is
    \begin{align*}
      P(G_i > G_{i+1}) &= p_{G_i}(3) \, P(G_{i+1} < 3) + p_{G_i}(1) \, P(G_{i+1} < 1) \\
                       &= \frac16 \paren{1 - \frac16} + \frac12 \, \frac13 = \frac{11}{36}.
    \end{align*}
    By symmetry, the probability that Bob wins at each round is also
    \[P(G_i < G_{i+1}) = \frac{11}{36}.\]
    This implies that the probability distribution of \(W_2\) is
    unaffected by the outcome of \(W_1\).  The probablity that either
    Alice or Bob wins at each round is
    \[P(G_i > G_{i+1} \textop{or} G_i < G_{i+1}) = P(G_i > G_{i+1}) + P(G_i < G_{i+1}) = \frac{11}{18}.\]
    So \(W_1\) and \(W_2\) both have the geometric distribution with
    parameter \(11/18\) and \(11/36\) respectively.

    Finally, the expectation to be found is
    \begin{align*}
      \E[N] &= \E[W_1 + W_2] \\
            &= \E[W_1] + \E[W_2] \\
            &= \frac{18}{11} + \frac{36}{11} = \frac{54}{11}.
    \end{align*}
  \end{enumerate}

\item \emph{Sum of a geometric number of independent geometric random variables}

  Let \(Y = X_1 + \dots + X_N\), where the random variable \(X_i\) are
  geometric with parameter \(p\), and \(N\) is geometric with
  parameter \(q\).  Assume that the random variables \(N\), \(X_1\),
  \(X_2\), \dots\ are independent.  Show that \(Y\) is geometric with
  parameter \(pq\).  \emph{Hint}: Interpret the various random variables in
  terms of a split Bernoulli process.

\item A train bridge is constructed across a wide river.  Trains
  arrive at the bridge according to a Poisson process of rate
  \(λ = 3\) per day.
  \begin{enumerate} \parasp
  \item If a train arrives on day \(0\), find the probability that there
    will be no trains on days \(1\), \(2\), and \(3\).

    Let \(X\) be the day of next train arrival, then the probabiltiy to be found is
    \[P(X > 3) = \int_3^{+\infty} 3e^{-3t} \dt = e^{-3t}\Bigr\vert_{+\infty}^3 = e^{-9}.\]

  \item Find the probability that the next train to arrive after the
    first train on day \(0\), takes more than \(3\) days to arrive.
    \[P(X > 3) = \int_3^{+\infty} 3e^{-3t} \dt = e^{-3t}\Bigr\vert_{+\infty}^3 = e^{-9}.\]

  \item Find the probability that no trains arrive in the first \(2\)
    days, but \(4\) trains arrive on the \(4\)th day.
    \[P(X > 2) P(4 \text{ train arrives on } 4 \text{th day})
      = \paren{\int_2^{+\infty} 3e^{-3t} \dt} \paren{\frac{3^4}{4!}e^{-3}}
      = \frac{3^4}{4!} e^{-9}.\]

  \item Find the probability that it takes more than \(2\) days for the
    \(5\)th train to arrive at the bridge.
    \begin{align*}
      \int_2^{+\infty} \frac{3^5 t^4}{5!} e^{-3t} \dt
      &= \sum_{i=0}^{4} \frac{6^i}{i!} e^{-6} \\
      &= (1 + 6 + 18 + 36 + 54) e^{-6} \\
      &= 115e^{-6}.
    \end{align*}
  \end{enumerate}
\end{enumerate}

\end{document}