\documentclass[a4paper]{ctexart}

\title{自考教材 02197(2018年版)勘误}
\author{赵磊}

\usepackage[T1]{fontenc}
\usepackage{textcomp}
\usepackage{mathtools,amssymb,amsthm}
\usepackage[hmargin=1in,vmargin=1in]{geometry}
\usepackage{graphicx,xcolor}
\usepackage[pdfusetitle]{hyperref}
\hypersetup{%
  colorlinks=true,
  urlcolor=[rgb]{0,0.2,0.6},
  linkcolor={.},
  bookmarksdepth=2}
\usepackage[group-minimum-digits=4]{siunitx}
\usepackage{float}
\usepackage{diagbox}

\makeatletter
\def\@maketitle{%
  \newpage
  \null
  \vskip 2em%
  \begin{center}%
    \let \footnote \thanks
    {\LARGE \@title \par}%
    \vskip 1.5em%
    {\large \@date}%
  \end{center}%
  \par
  \vskip 1.5em}
\def\hhline{%
  \noalign{\ifnum0=`}\fi\hrule \@height 2\arrayrulewidth \futurelet
   \reserved@a\@xhline}
\makeatother

\frenchspacing

\newcommand*{\Fn}[1]{\mathop{\relax #1}\nolimits}
\DeclarePairedDelimiterXPP{\pnorm}[1]{\Fn{\Phi}}{\lparen}{\rparen}{}{#1}
\newcommand\SetSymbol[1][]{%
  \nonscript\:#1\vert
  \allowbreak
  \nonscript\:
  \mathopen{}}
\let\Pr\relax
\DeclarePairedDelimiterXPP{\Pr}[1]{\Fn{P}}{\lbrace}{\rbrace}{}{%
  \renewcommand{\mid}{\SetSymbol[\delimsize]}#1}
\DeclarePairedDelimiter{\abs}{\lvert}{\rvert}
\DeclarePairedDelimiterX{\paren}[1]{\lparen}{\rparen}{%
  \renewcommand{\mid}{\SetSymbol[\delimsize]}#1}
\DeclarePairedDelimiterXPP{\E}[1]{\Fn{E}}{\lparen}{\rparen}{}{%
  \renewcommand{\mid}{\SetSymbol[\delimsize]}#1}
\DeclarePairedDelimiterXPP{\D}[1]{\Fn{D}}{\lparen}{\rparen}{}{%
  \renewcommand{\mid}{\SetSymbol[\delimsize]}#1}
\DeclarePairedDelimiterX{\brkt}[1]{\lbrack}{\rbrack}{%
  \renewcommand{\mid}{\SetSymbol[\delimsize]}#1}
\DeclarePairedDelimiterXPP{\Cov}[1]{\qopname\relax o{Cov}}{\lparen}{\rparen}{}{%
  \renewcommand{\mid}{\SetSymbol[\delimsize]}#1}
\newcommand*{\diff}{\mathop{}\!\mathit{d}}
%\newcommand*{\diff}{\mathop{}\!\mathrm{d}}
\newcommand*{\dx}{\diff x}
\newcommand*{\dy}{\diff y}
\newcommand*{\dd}[2][]{\frac{\diff#1}{\diff#2}}

% \usepackage[lite,subscriptcorrection,nofontinfo]{mtpro2}
\usepackage{fontspec}

\defaultfontfeatures{Ligatures=TeX}
\setmainfont{Palatino Linotype}
% \setmonofont{Source Code Pro}
% \usepackage[integrals]{wasysym}
% \usepackage{fontawesome}

\ctexset{punct=CCT,proofname=证明\nopunct}
% \xeCJKsetup{CJKecglue=\,}
\xeCJKDeclareCharClass{Default}{`—}
% \ltjsetparameter{xkanjiskip={0.13\zw plus 1pt minus 1pt}}
% \ltjsetparameter{jaxspmode={"FF08,inhibit}}
\setCJKmainfont{Songti SC}[
  BoldFont = * Black,
  ItalicFont = Heiti SC Medium
]

\usepackage[math-style=TeX]{unicode-math}
\setmathfont{TeX Gyre Pagella Math}

\usepackage{microtype}

\AtBeginDocument{%
  \let\leq\leqslant
  \let\le\leq
  \let\geq\geqslant
  \let\ge\geq}

\begin{document}
\maketitle

第35页,习题1.1第3题的(7),答案说(1)(4)(5)(6)(8)成立,其他不成立.
实际上,(7)也是成立的.

\begin{proof}
  在给定的条件下,假设\(BC \ne \emptyset\),这就是说存在一个\(x \in BC\).
  那么\(x \in B\)且\(x \in C\).  因为\(C \subset A\),所以\(x \in A\).
  可见\(x\)同时属于\(A\)和\(B\),所以有\(x \in AB\).  这也就是说\(AB\)非空,
  但是这和我们的条件矛盾,所以必然有\(BC = \emptyset\).
\end{proof}

第60页,习题2.1的第2题,答案分子分母颠倒了.
\[
  \frac{1}{2c} + \frac{3}{4c} + \frac{5}{8c} + \frac{7}{16c} = 1
\]
解得\(c = 37/16\).  答案给的是\(16/37\).

第73页,第三章第3节例8的(3),倒数第二行出现了印刷错误,那一行最后应该是加
上\(1 - \pnorm{2.25}\).  整个计算过程如下
\begin{align*}
  \Pr{\abs{X} \ge 3}
    &= 1 - \Pr{\abs{X} < 3} = 1 - \Pr{-3 < X < 3} \\
    &= 1 - F(3) + F(-3)
      = 1 - \pnorm[\bigg]{\frac{3-1.5}{2}} + \pnorm[\bigg]{\frac{-3-1.5}{2}} \\
    &= 1 - \pnorm{0.75} + \pnorm{-2.25} \\
    &= 1 - \pnorm{0.75} + 1 - \pnorm{2.25} \\
    &= 1 - \num{0.7734} + 1 - \num{0.9878} = \num{0.2388}.
\end{align*}

第82页,自测题2选择题的第11题的D选项应该是\(\frac12 \, f_X\paren*{-\frac{y}{2}}\),
中间少了一个负号,这可以由第77页的定理1直接得到,或者从分布函数的定义出发,然后求导
得到.

第82页,自测题2填空题的第8题出现了印刷错误,应该是分布函数为\(F(x) = \dots\),
其中\(F\)应为大写。

第83页,自测题2填空题的第13题出现了印刷错误,应该是\(X \sim N(0, 1)\),而不
是\(X \sim N(0.1)\).

第86页,第三章第1节的例3,计算中使用的数据和分布律中给出的不一致.  以分布律
中给出的为准,(1)的最后计算应该是\(0.1 + 0.1 + 0.3 = 0.5\),(3)的最后计
算应该是\(0.1 + 0.1 = 0.2\),(4)的最后计算应该是\(0.1 + 0.25 = 0.35\).
当然,我们也可以将分布律改成表\ref{tab:1}的样子,这时(1)(3)(4)都没有
问题了,但是(2)的最后计算就要改成\(0.15 + 0.2 + 0.15 + 0 = 0.5\)了.
\begin{table}[H]
  \centering
  \begin{tabular}{c|ccc}
    \hhline
    \diagbox{\(X\)}{\(Y\)}
      & 1    & 2    & 3 \\
    \hline
    0 & 0.15 & 0.15 & 0.2 \\
    1 & 0.2  & 0    & 0.3 \\
    \hhline
  \end{tabular}
  \caption{修改后的分布律}
  \label{tab:1}
\end{table}

第101页,第三章第2节的定义10,边缘分布函数定义的右边的第\(k\)个不等式,最好
使用小于等于号,如果讨论的是连续随机变量,自然不影响,但如果是离散随机变量
或者混合随机变量,那么小于号就不合适了.

第102页,第三章第2节的例10,两个随机变量在\([1, 3]\)上的概率密度都应该
是\(1/2\),而不是\(2\).

第103页,习题3.2的第3题和第101页的例8重复了,那么就没什么练习价值了.

第120页,第四章第1节例13的(2),过程中积分的边界确定错了,但是表达
式\((x+y)/4\)是对称的,用\(x\)替换\(y\),用\(y\)替换\(x\),会得到原表达
式,所以算出来的结果还是对的.  正确的边界应该是
\begin{align*}
  \E{X + Y}
    &= \int_{-\infty}^{+\infty} \int_{-\infty}^{+\infty} \frac14 (x+y) \dx\dy \\
    &= \frac14 \int_0^2 \int_{-1}^1 (x+y) \dx\dy \\
    &= \frac14 \int_0^2 \int_{-1}^1 x \dx\dy + \frac14 \int_0^2 \int_{-1}^1 y \dx\dy \\
    &= \frac14 \paren*{\int_0^2 y \dy} \paren*{\int_{-1}^1 \dx} \\
    &= \frac14 \times \frac{2^2}{2} \times 2 = 1.
\end{align*}

第124页,第四章第2节例4的最后一个式子应该是
\[
  \D{X} = \E[\big]{X^2} - \brkt[\big]{\E{X}}^2 = \frac16 - 0^2 = \frac16.
\]

第136页,第四章第3节的\emph{注意},不等号应该是等于号.  事实上,若\(\Cov{X, Y} = 0\),
则\(X\)与\(Y\)不一定相互独立;若\(\Cov{X, Y} \ne 0\),则\(X\)与\(Y\)一定不相互独立.

第144页,自测题4的第13—16题,这几道题中离散二字可以去掉,其中第13、15、16题,
离散二字应该改成连续或者直接去掉.

第145页,自测题4填空题的第5题和第9题重复了.

第153页,第五章第3节,该页第一行中的表达式应该是\(N(n\mu, n\sigma^2)\).

第153页,第五章第3节的例2,参数\(\lambda\)应该是\(1/100\).

第174页,第六章的小结中,样本方差的期望值应该是\(\E{S^2} = \sigma^2\).

第180页,第七章第1节的例8,最后的方程应该是
\[
  \dd[\ln L(\lambda)]\lambda = \frac{1}{\lambda} \sum_{i=1}^n x_i - n = 0.
\]

第186页,第七章第3节的例1,中间有个印刷错误,\(\sqrt n\)被印成了\(\sqrt\pi\).

第190页,第七章第3节的例6,有个式子应该是\(\chi_{\alpha/2}^2 (8) = \num{17.5345}\).

第191页,第七章第3节例7的(1),出现了计算错误,最后的置信区间应该
是\([15.233, 15.615]\).

第201页,第八章第2节的最后一段,有一个错别字,应该是显著性检验.
\end{document}

% Local Variables:
% TeX-engine: xetex
% End:
